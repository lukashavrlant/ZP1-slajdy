\documentclass{beamer}
\usetheme[pageofpages=z,
          bullet=circle,
          titleline=true,
          alternativetitlepage=true,
          titlepagelogo=uplogo,
          ]{UVT}

\usepackage[utf8x]{inputenc}
\usepackage{czech}
\usepackage{minted}
\usepackage{color}
\usepackage{amssymb}

% \usepackage{url}


\newenvironment{itemizex}%
  {\large \begin{itemize}%
    \setlength{\itemsep}{8pt}%
    \setlength{\parskip}{8pt}}%
  {\end{itemize}}

\newenvironment{enumeratex}%
  {\large \begin{enumerate}%
    \setlength{\itemsep}{6pt}%
    \setlength{\parskip}{6pt}}%
  {\end{enumerate}}


\title{Základy programování 1 (KMI/ZP1 a KMI/UP1)}
\subtitle{Cvičení 7: Opakování látky}
\author{Lukáš Havrlant}
\date{24. října 2012}
\institute{Univerzita Palackého}

\begin{document}

\begin{frame}[t,plain]
\titlepage
\end{frame}



\begin{frame}[t,fragile]\frametitle{Z čeho se skládá zdrojový soubor} 
\begin{minted}[linenos=true]{c} 
#include <stdio.h>
#define POZDRAV "Nashledanou!\n"

int status = 0; 

void slusne_pozdrav(char[]);

void slusne_pozdrav(char pozdrav[]) { 
    printf("%s", pozdrav); 
}

int main() {
    slusne_pozdrav(POZDRAV);
    return status;     
}
\end{minted}
\end{frame}

\begin{frame}[t,fragile]\frametitle{Rozdíl oproti Scheme} 
\begin{minted}[linenos=true]{scheme} 
(define pozdrav 'nasledanou)
(define slusne-pozdrav
    (lambda (x)
        (print x)))

(slusne-pozdrav pozdrav)
\end{minted}

\begin{itemizex}
    \item Ve Schemu nepotřebujeme \texttt{main} funkci. 
    \item A nepotřebujeme includovat hlavičkové soubory kvůli printu
\end{itemizex}
\end{frame}


\begin{frame}[t,fragile]\frametitle{Proměnné} 
\begin{minted}[linenos=true]{c} 
int main() {
    int x = 0;          // Cele cislo
    int y;              // Cele cislo, ale neinicializovane
    double q = 5.123;   // Desetinne cislo
    char z = '!';       // Znak 

    printf("x: %i, y: %i\n", x, y); 
    // x: 0, y: 1481518256
    // x: 0, y: 1588624560

    printf("Znak: %c, ciselny kod: %i\n", z, z);
    // Znak: !, ciselny kod: 33
    return 0;
}
\end{minted}
\end{frame}


\begin{frame}[t,fragile]\frametitle{Rozdíl oproti Scheme} 
\begin{minted}[linenos=true]{scheme} 
(let ((a 0)
      (b 12.4465)
      (c 1232354576785632452345234123534)
      (d 'symbol)
      (e '(1 2 3 4)))
  c)
\end{minted}
\end{frame}


% \begin{frame}[t,fragile]\frametitle{Logické operátory} 
% \begin{itemizex}
%     \item Nula je \uv{nepravda}. Vše ostatní je \uv{pravda}.
% \end{itemizex}
% \begin{minted}[linenos=true]{c} 
% int x;
% x = 6 && 0;
% x = 6 || 0;       
% x = !6;           
% x = !(7 && 10);  
% x = 6 < !10;     
% x = (6 == 6);
% x = (6 != 7);  

% int y = 10;     
% x = (y == 10);  
% x = (y = 15);   
% \end{minted}
% \end{frame}


\begin{frame}[t,fragile]\frametitle{Logické operátory v Céčku ve Scheme} 
\begin{minted}[linenos=true]{c} 
x = 6 && 0;
x = 6 || 0;
x = 0 || (5 == 0) || 10 || 0;
x = !6;
x = !(0 &&);
\end{minted}

\begin{minted}[linenos=true]{scheme} 
(and 6 0)
(and 6 #f)
(or 6 #f)
(or #f (= 5 0) 10 #f)
(not 6)
(not (and 0))
(and)
(or)
\end{minted}
\end{frame}


\begin{frame}[t,fragile]\frametitle{Větvení programu} 
\begin{minted}[linenos=true]{c} 
int main() {
    int a = 10, b = 20;
    if (2 * a < b) {
        printf("...");
    } else {
        printf("!!!");
    }

    if (a = b) {
        printf("a: %i", a); 
    } else {
        printf("a: %i", a); 
    }
    return 0;
}
\end{minted}
\end{frame}


\begin{frame}[t,fragile]\frametitle{Cyklus while} 
\begin{minted}[linenos=true]{c} 
int main() {
    int i = 0;
    while (i < 5) {
        printf("%i, ", i);
        i++;
    } // 0, 1, 2, 3, 4

    return 0;
}
\end{minted}
\end{frame}


\begin{frame}[t,fragile]\frametitle{Jiný příklad} 
\begin{minted}[linenos=true]{c} 
int main() {
    int i = 1, mocnina;
    while (i * i < 100) {
        printf("%i, ", i * i);
        i++;
    } // 1, 4, 9, 16, 25, 36, 49, 64, 81,

    return 0;
}
\end{minted}
\end{frame}


\begin{frame}[t,fragile]\frametitle{Inkrementovat můžeme jak chceme} 
\begin{minted}[linenos=true]{c} 
int main() {
    int i = 1;
    while (i < 100) {
        printf("%i, ", i);
        i += i;
    } // 1, 2, 4, 8, 16, 32, 64, 


    for (i = 1; i < 100; i += i)
        printf("%i, ", i);
    // 1, 2, 4, 8, 16, 32, 64, 

    return 0;
}
\end{minted}
\end{frame}


\begin{frame}[t,fragile]\frametitle{Trojúhelník} 
\begin{minted}[linenos=true]{c} 
int main() {
    int i, j, delka = 4;

    for (i = 1; i <= delka; i++) {
        for (j = 0; j < i; j++) {
            printf("*");
        }
        printf("\n");
    }
}
// *
// **
// ***
// ****
\end{minted}
\end{frame}


\begin{frame}[t,fragile]\frametitle{Co je to jednorozměrné pole}
\begin{itemizex}
    \item Je to tabulka s jedním řádkem a $x$ sloupci. Délka pole = $x$ = počet sloupců.
    \item Příklad: pole o délce 5:
    \item 
\begin{tabular}{|c|c|c|c|c|}
\hline 1& 2 & 4 & 8 &16 \\\hline
\end{tabular}
    \item Každá buňka obsahuje jeden prvek (jedno číslo).
    \item Prvky pole jsou očíslovány:
    \item 
\begin{tabular}{|c|c|c|c|c|}
\hline 1& 2 & 4 & 8 &16 \\\hline
\end{tabular}\\
\begin{tabular}{ccccc}
 0&1&2&3&4
\end{tabular}
\end{itemizex}

\end{frame}


\begin{frame}[t,fragile]\frametitle{Jak přistupovat k prvkům pole} 
\begin{minted}[linenos=true]{c} 
int pole[] = {14, 10, -5, 14, 58, 69, 16};
\end{minted}

\begin{tabular}{|c|c|c|c|c|c|c|}
\hline 14& 10 & -5 & 14 &58 &69& 16 \\\hline
\end{tabular}\\
\begin{tabular}{ccccccc}
 \,0\,\,\,&1\,\,\,\,&2\,\,\,&3\,\,\,&4\,\,&5\,\,&6
\end{tabular}
\vskip 1cm
\begin{minted}[linenos=true]{c} 
int cislo = pole[5];
\end{minted}

\begin{tabular}{|c|c|c|c|c|c|c|}
\hline 14& 10 & -5 & 14 &58 &{\color{red} 69}& 16 \\\hline
\end{tabular}\\
\begin{tabular}{ccccccc}
 \,0\,\,\,&1\,\,\,\,&2\,\,\,&3\,\,\,&4\,\,&{\color{red}5}\,\,&6
\end{tabular}

\end{frame}


\begin{frame}[t,fragile]\frametitle{Jak zapisovat do pole} 
\begin{minted}[linenos=true]{c} 
int pole[] = {14, 10, -5, 14, 58, 69, 16};
pole[1] = 47;
\end{minted}

\begin{center}
\begin{tabular}{|c|c|c|c|c|c|c|}
\hline 14& {\color{red}10} & -5 & 14 &58 &69& 16 \\\hline
\end{tabular}\\
\begin{tabular}{ccccccc}
 \,0\,\,\,&{\color{red}1}\,\,\,\,&2\,\,\,&3\,\,\,&4\,\,&5\,\,&6
\end{tabular}
$$
\downarrow
$$
\begin{tabular}{|c|c|c|c|c|c|c|}
\hline 14& {\color{red}47} & -5 & 14 &58 &69& 16 \\\hline
\end{tabular}\\
\begin{tabular}{ccccccc}
 \,0\,\,\,&{\color{red}1}\,\,\,\,&2\,\,\,&3\,\,\,&4\,\,&5\,\,&6
\end{tabular}
\end{center}

\begin{minted}[linenos=true]{c} 
int cislo = pole[1]; // cislo = 47
\end{minted}
\end{frame}


\begin{frame}[t,fragile]\frametitle{Definice pole} 
\begin{minted}[linenos=true]{c} 
int pole1[5];
int pole2[] = {2, 7, 9, 14, -7, -7};
int pole3[10] = {-5, 81};
\end{minted}

\vskip 5mm
pole1:
\begin{tabular}{|p{4mm}|p{4mm}|p{4mm}|p{4mm}|p{4mm}|}
\hline & & & & \\\hline
\end{tabular}

\vskip 5mm
pole2:
\begin{tabular}{|p{4mm}|p{4mm}|p{4mm}|p{4mm}|p{4mm}|p{4mm}|}
\hline 2& 7& 9& 14& -7 & -7\\\hline
\end{tabular}

\vskip 5mm
pole3:
\begin{tabular}{|p{4mm}|p{4mm}|p{4mm}|p{4mm}|p{4mm}|p{4mm}|p{4mm}|p{4mm}|p{4mm}|p{4mm}|}
\hline -5&81&0&0&0&0&0&0&0&0 \\\hline
\end{tabular}
\end{frame}



\begin{frame}[t,fragile]\frametitle{Procházení pole} 
\begin{minted}[linenos=true]{c} 
int main() {
    int delka = 6, i, soucet = 0;
    int vek[] = {15, 18, 64, 34, 42, 2};

    for (i = 0; i < delka; i++)
        printf("%i ", vek[i]);
    // Vypise: 15 18 64 34 42 2 

    for (i = 0; i < delka; i++) 
        soucet += vek[i];
    printf("%i", soucet);
    // 175

    return 0;     
}
\end{minted}
\end{frame}


\begin{frame}[t,fragile]\frametitle{Změna prvků v poli} 
\begin{minted}[linenos=true]{c} 
int main() {
    int delka = 5, i;
    int pole[] = {6, 0, 7, 1, 10};

    for (i = 0; i < delka; i++) 
        pole[i] = pole[i] * 2;

    return 0;     
}
\end{minted}
\begin{center}
\begin{tabular}{|c|c|c|c|c|}
\hline 6&0&7&1&10 \\\hline
\end{tabular}
$$
\downarrow
$$
\begin{tabular}{|c|c|c|c|c|}
\hline 12&0&14&2&20 \\\hline
\end{tabular}
\end{center}

\end{frame}


\begin{frame}[t,fragile]\frametitle{Posunutí pole} 
\begin{minted}[linenos=true]{c} 
int delka = 6, i, temp;
int pole[] = {4, 5, 6, 7, 8, 9};

temp = pole[delka-1];
for (i = delka - 1; i > 0; i--) 
    pole[i] = pole[i - 1];
pole[0] = temp;
\end{minted}

\begin{center}
\begin{tabular}{|c|c|c|c|c|c|}
\hline 4&5&6&7&8&9 \\\hline
\end{tabular}
$$
\downarrow
$$
\begin{tabular}{|c|c|c|c|c|c|}
\hline 9&4&5&6&7&8 \\\hline
\end{tabular}
\end{center}
\end{frame}


\begin{frame}[t,fragile]\frametitle{Řetězec} 
\begin{minted}[linenos=true]{c} 
int main() {
    char pozdrav1[] = "ahoj";
    char pozdrav2[] = {'a', 'h', 'o', 'j', '\0'};
    printf("%s\n", pozdrav1); // ahoj
    printf("%s\n", pozdrav2); // ahoj
    return 0;     
}
\end{minted}

pozdrav1, pozdrav2: \\
\begin{center}
\begin{tabular}{|c|c|c|c|c|}
\hline 'a'&'h'&'o'&'j'&'\textbackslash0' \\\hline
\end{tabular}
\end{center}
\vskip 5mm
Případně jako číselné pole:\\
\begin{center}
\begin{tabular}{|c|c|c|c|c|}
\hline 97 &104 &111 &106 &0 \\\hline
\end{tabular}
\end{center}
\end{frame}


\begin{frame}[t,fragile]\frametitle{Rozdíl mezi číselným polem a řetězcem} 
\begin{itemizex}
    \item U klasického číselného pole není v paměti žádná \uv{zarážka}
    \item Pokud neznáme délku pole, nepoznáme, jestli $65$ patří k našemu poli nebo ne.
    \item U řetězce zarážku máme.
    \begin{minted}[linenos=true]{c} 
int cisla[] = {3, 6, 9, 12, 15};
char skupina[] = "Burzum";
    \end{minted}
\end{itemizex}
\begin{center}
cisla $\longrightarrow\,$\begin{tabular}{|c|c|c|c|c|c|c|c|c|c|c|}
\hline 3& 6& 9& 12& 15 &18&-457&145&964&74&-4 \\\hline
\end{tabular}
\vskip 5mm
skupina $\longrightarrow\,$\begin{tabular}{|c|c|c|c|c|c|c|c|c|c|c|}
\hline 'B'&'u'&'r'&'z'&'u'&'m'&'\textbackslash0'&'x'&'y'&'!' \\\hline
\end{tabular}
\end{center}
\end{frame}



\begin{frame}[t,fragile]\frametitle{Jak procházet řetězec} 
\begin{minted}[linenos=true]{c} 
int main() {
    int i, delka = 5;
    int cisla[] = {3, 6, 9, 12, 15};
    char skupina[] = "Burzum";

    for (i = 0; i < delka; i++)
        printf("%i, ", cisla[i]);
    // 3, 6, 9, 12, 15

    for (i = 0; skupina[i] != '\0'; i++) 
        printf("%c, ", skupina[i]);
    // B, u, r, z, u, m, 

    return 0;     
}
\end{minted}
\end{frame}


\begin{frame}[t,fragile]\frametitle{S řetězcem můžeme pracovat jako s číselným polem} 
\begin{minted}[linenos=true]{c} 
int main() {
    int i;
    char retezec[] = "abcdef";

    for (i = 0; retezec[i] != '\0'; i++)
        retezec[i] += 2;

    printf("%s\n", retezec);
    // cdefgh

    return 0;     
}
\end{minted}
\end{frame}


\begin{frame}[t,fragile]\frametitle{Funkce} 
\begin{itemizex}
    \item Obecná definice:
    \begin{minted}[linenos=true]{c} 
typ nazev(typ1 nazev1, typ2 nazev2, ...) {
    // telo funkce
}
    \end{minted}
    \item Příklad:
    \begin{minted}[linenos=true]{c} 
double prumer(int a, int b, int c) {
    return (a + b + c) / 3.0;
}
    \end{minted}
\end{itemizex}
\end{frame}


\begin{frame}[t,fragile]\frametitle{Souvislosti s matematikou} 
    \begin{itemize}
        \item Funkce $\sin$ je zobrazení ${\color{red}\mathbb{R}}\longrightarrow{\color{blue}\mathbb{R}}$. Nebo relace mezi ${\color{red}\mathbb{R}}$ a ${\color{blue}\mathbb{R}}$. 
        % \item Funkce \texttt{delka_retezce} je zobrazení ${\color{red}\mathbb{R}}\longrightarrow{\color{blue}\mathbb{R}}$. Případně relace mezi ${\color{red}\mathbb{R}}$ a ${\color{blue}\mathbb{R}}$.
        \item {\color{red} Definiční obor} $\longrightarrow$ {\color{blue} Kodefiniční obor}, $\sin ({\color{red}x}) = {\color{blue}y}$. 
        \item {\tiny Pozn.: \uv{Kodefiniční obor} se v češtině nepoužívá, je to z anglického \uv{codomain}.}
        \item {\color{blue} Obor hodnot} = podmnožina kodefiničního oboru: $R(\sin)=\left<-1, 1\right>$
        \item Tedy platí: $\sin({\color{red}\pi}) = {\color{blue}0}$, případně $\left<{\color{red}\pi}, {\color{blue}0}\right>\in\sin$.
        \item \texttt{{\color{blue}double} sin({\color{red}double x}) \{ ... \}}
        \item Vstupem funkce je prvek z definičního oboru, výstupem je prvek z kodefiničního oboru (přesněji z oboru hodnot).
        \item Výstupem funkce je vždy jedna hodnota: Kolik je $\sqrt{9}$?
        \item \uv{Matematická funkce} vždy vrací pro stejné parametry stejný výsledek. Funkce v C nemusí (globální proměnné, soubory)
    \end{itemize}
\end{frame}


\begin{frame}[t,fragile]\frametitle{Souvislosti se Scheme} 
\begin{itemize}
    \item C:
\begin{minted}[linenos=true]{c} 
int na2(int x) {
    return x * x;
}

int main() {
    na2(6);
}
\end{minted}
    \item Scheme:
\begin{minted}[linenos=true]{scheme} 
(define na2
  (lambda (x)
    (* x x)))

(na2 6)
\end{minted}
\end{itemize}
\end{frame}


\begin{frame}[t,fragile]\frametitle{Na čtvrtou} 
\begin{itemizex}
    \item Matematicky:
$$
x^4 = (x^2)^2
$$
    \item C:
\begin{minted}[linenos=true]{c} 
int na4(int x) {
    return na2(na2(x));
}
\end{minted}
    \item Scheme:
\begin{minted}[linenos=true]{scheme} 
(define na4
  (lambda (x)
    (na2 (na2 x))))
\end{minted}
\end{itemizex}
\end{frame}



\begin{frame}[t,fragile]\frametitle{Řešení kvadratické rovnice} 
\begin{minted}[linenos=true]{c} 
double diskriminant(double a, double b, double c) {
    return na2(b) - (4 * a * c);
}
double reseni_rovnice(double a, double b, double odm_diskr) {
    return (-b + odm_diskr) / (2 * a);
}
void kv_rovnice(double a, double b, double c) {
    double diskr = diskriminant(a, b, c); // diskr = 1
    double odm_diskr = sqrt(diskr); 
    printf("x1: %g\n", reseni_rovnice(a, b, odm_diskr)); 
    printf("x2: %g\n", reseni_rovnice(a, b, -odm_diskr)); 
}
int main() { // x^2 + 7x + 12 = 0
    kv_rovnice(1, 7, 12);  
}
\end{minted}
\end{frame}



\begin{frame}[t,fragile]{Úlohy}
\begin{center}
\vskip 1cm
{\Large Seznam úloh je na \url{http://KMIup1.jdem.cz}}
\vskip 2cm
\url{http://tux.inf.upol.cz/~havrlant/}
\end{center}
\end{frame}


\end{document}