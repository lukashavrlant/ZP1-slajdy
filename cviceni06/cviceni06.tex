\documentclass{beamer}
\usetheme[pageofpages=z,
          bullet=circle,
          titleline=true,
          alternativetitlepage=true,
          titlepagelogo=uplogo,
          ]{UVT}

\usepackage[utf8x]{inputenc}
\usepackage{czech}
\usepackage{minted}
% \usepackage{url}


\newenvironment{itemizex}%
  {\large \begin{itemize}%
    \setlength{\itemsep}{8pt}%
    \setlength{\parskip}{8pt}}%
  {\end{itemize}}

\newenvironment{enumeratex}%
  {\large \begin{enumerate}%
    \setlength{\itemsep}{6pt}%
    \setlength{\parskip}{6pt}}%
  {\end{enumerate}}


\title{Základy programování 1 (KMI/ZP1 a KMI/UP1)}
\subtitle{Cvičení 6: Funkce}
\author{Lukáš Havrlant}
\date{24. října 2012}
\institute{Univerzita Palackého}

\begin{document}

\begin{frame}[t,plain]
\titlepage
\end{frame}


\begin{frame}[t,fragile]\frametitle{Co je to funkce} 
    \begin{itemizex}
        \item Část programu, kterou je možné opakovaně volat z různých míst kódu.
        \item Funkce zpřehledňují programy.
        \item Funkce jsme už používali: \texttt{printf}, \texttt{main}
    \end{itemizex}
\end{frame}


\begin{frame}[t,fragile]\frametitle{Definice jednoduché funkce} 
    \begin{itemizex}
        \item Syntaxe:
        \begin{minted}[linenos=true]{c} 
void nazev_funkce() {
    // prikazy...
}
        \end{minted}
        \item Funkce definujeme mimo \texttt{main} funkci.
        \item Příklad:
        \begin{minted}[linenos=true]{c} 
void novy_radek() {
  printf("\n");
}
        \end{minted}
    \end{itemizex}
\end{frame}


\begin{frame}[t,fragile]\frametitle{Volání funkce} 
    \begin{itemize}
        \item Pokud chceme zavolat funkci, použijeme k tomu operátor kulaté závorky \texttt{()}
        \begin{minted}[linenos=true]{c} 
// Toto je definice funkce novy_radek
void novy_radek() {
    printf("\n");
}

int main() {
    printf("Ahoj svete!"); 
    // Tady volame funkci novy radek
    novy_radek(); 
    return 0;         
}
// Vytiskne se: Ahoj svete!\n
        \end{minted}
    \end{itemize}
\end{frame}


\begin{frame}[t,fragile]\frametitle{Funkci můžeme použít vícekrát} 
    \begin{itemize}
        \item Jednou definovanou funkci můžeme použít kolikrát chceme:
        \begin{minted}[linenos=true]{c} 
void novy_radek() {
    printf("\n");
}

int main() {
    novy_radek();
    printf("Ahoj ");
    novy_radek();
    printf("svete!");
    novy_radek();
    return 0;
}
        \end{minted}
    \end{itemize}
\end{frame}


\begin{frame}[t,fragile]\frametitle{Parametry funkce} 
    \begin{itemizex}
        \item Co když nechceme vytisknout jen jeden řádek, ale obecně $n$ řádků?
        \item Jak bychom napsali funkci, která by vypsala větší ze dvou čísel? 
        \item $\longrightarrow$ pomocí parametrů funkce.
        \item Parametry funkce nám umožňují předat funkci nějaké hodnoty.
        \begin{minted}[linenos=true]{c} 
void nazev_funkce(typ nazev1, typ nazev2, ...) { 
    // prikazy... 
}
        \end{minted}
    \end{itemizex}
\end{frame}



\begin{frame}[t,fragile]\frametitle{Příklad: Vypiš maximum ze dvou čísel} 
    \begin{itemizex}
        \item Funkce, která vypíše větší z předaných čísel:
        \begin{minted}[linenos=true]{c} 
void vypis_vetsi(int a, int b) {
    int max = a > b ? a : b;
    printf("Vetsi: %i", max);
}

int main() {
    int x = -8, y = -51;
    vypis_vetsi(2, 6); // Vetsi: 6
    vypis_vetsi(x, y); // Vetsi: -8
    vypis_vetsi(x + 8, vypis_vetsi(2, 6)); Vetsi: 6
    return 0;
}
        \end{minted}
    \end{itemizex}
\end{frame}


\begin{frame}[t,fragile]\frametitle{Příklad: Vypiš pole} 
    \begin{itemizex}
        \item Funkce na vypsání pole čísel:
        \begin{minted}[linenos=true]{c} 
void vypis_pole_cisel(int[] pole, int delka) {
    int i;
    for (i = 0; i < delka; i++) {
        printf("%i ", pole[i]);
    }
}

int main() {
    int cisla[] = {1, 2, -7, 14};
    vypis_pole_cisel(cisla, 4); // 1 2 -7 14 
    return 0;
}
        \end{minted}
    \end{itemizex}
\end{frame}


\begin{frame}[t,fragile]\frametitle{Návratová hodnota funkce} 
    \begin{itemize}
        \item Co když nechceme vypsat maximum dvou čísel, ale jen ho zjistit? 
        \item Tedy chceme, aby nám funkce nějak vrátila větší ze dvou čísel.
        \item Takto by se to mělo chovat: 
        \begin{minted}[linenos=true]{c} 
int vetsi = vetsi_cislo(5, 7);
        \end{minted}
        \item K tomu slouží příkaz \texttt{return} 
        \item V definici musíme uvést, jaký typ funkce vrací. 
        \item Pokud funkce nevrací nic, použijeme typ \texttt{void}. Syntaxe:
        \begin{minted}[linenos=true]{c} 
typ nazev_funkce(/* parametry */) {
    // prikazy vcetne prikazu return
}
        \end{minted}
    \end{itemize}
\end{frame}


\begin{frame}[t,fragile]\frametitle{Příklad: Větší ze dvou čísel} 
    \begin{itemize}
        \item Funkce, která vrací větší ze dvou čísel:
        \begin{minted}[linenos=true]{c} 
int vetsi_cislo(int a, int b) {
    int max = a > b ? a : b;
    return max;
}
        \end{minted}
        \item Výsledkem volání funkce bude hodnota vrácená příkazem \texttt{return}
        \item S tou můžeme dále pracovat:
        \begin{minted}[linenos=true]{c} 
int main() {
    int vetsi = vetsi_cislo(5, 7); // vetsi = 7
    printf("vetsi: %i", vetsi_cislo(0, -4)); // vetsi: 0
    return 0;
}
        \end{minted}
    \end{itemize}
\end{frame}


\begin{frame}[t,fragile]\frametitle{Příklad: Počet malých písmen v řetězci} 
    \begin{itemize}
        \item Funkce, která vrátí počet malých písmen v předaném řetězci
        \begin{minted}[linenos=true]{c} 
int pocet_malych_pismen(char[] retezec) {
    char znak;
    int i, pocet = 0;
    for (i = 0; znak = retezec[i]; i++) {
        if (znak >= 'a' && znak <= 'z') 
            pocet++;
    }
    return pocet;
}
int main() {
    printf("pocet: %i", pocet_malych_pismen("abcDEF"));
    return 0;
}
        \end{minted}
    \end{itemize}
\end{frame}


\begin{frame}[t,fragile]\frametitle{Více příkazů return} 
    \begin{itemizex}
        \item Funkce může obsahovat více příkazů \texttt{return}.
        \item Ve chvíli, kdy se narazí na první \texttt{return}, funkce skončí a vrátí hodnotu.
        \begin{minted}[linenos=true]{c} 
int obsahuje_cislici(char[] retezec) {
    char znak;
    int i;
    for (i = 0; znak = retezec[i]; i++) {
        if (znak >= '0' && znak <= '9') 
            return 1;
    }
    return 0;
}
        \end{minted}
    \end{itemizex}
\end{frame}


\begin{frame}[t,fragile]\frametitle{Cokoliv za returnem bude ignorováno} 
    \begin{itemizex}
        \item Tajemství funkce se nikdy nedovíme, \texttt{return} ukončí funkci dříve než nám to funkce stihne říci:
        \begin{minted}[linenos=true]{c} 
char rekni_tajemstvi() {
    return 'x';
    printf("Zboznuji Justina Biebera!\n");
}

int main() {
    char z = rekni_tajemstvi();
    return 0;
}
        \end{minted}
    \end{itemizex}
\end{frame}


\begin{frame}[t,fragile]\frametitle{Nemůžeme ale vrátit pole} 
    \begin{itemizex}
        \item Tohle nemůžeme udělat:
        \begin{minted}[linenos=true]{c} 
int[] vrat_pole() {
    int pole[] = {1, 2, 3};
    return pole;
}
        \end{minted}
        \item  O tom ale příště.
    \end{itemizex}
\end{frame}


\begin{frame}[t,fragile]\frametitle{Vrácenou hodnotu můžeme ignorovat} 
    \begin{itemize}
        \item Pokud funkce vrací nějakou hodnotu, nemusíme ji nijak používat
        \begin{minted}[linenos=true]{c} 
int soucet(int a, int b) {
    int soucet = a + b;
    printf("Soucet a + b = %i", soucet);
    return soucet;
}

int main() {
    int a;
    soucet(4, 5);
    a = soucet(10, 15);
    return 0;
}
        \end{minted}
    \end{itemize}
\end{frame}


\begin{frame}[t,fragile]\frametitle{Rozsah platnosti lokálních proměnných} 
    \begin{itemizex}
        \item Parametry a lokální (interní) proměnné funkce existují pouze v rámci této funkce
        \item Při vstupu do funkce je proměnná vytvořena
        \item Při ukončení funkce je proměnná odstraněna
        \item Platí to obecně pro jakýkoliv blok (např. cyklus)
    \end{itemizex}
\end{frame}

\begin{frame}[t,fragile]\frametitle{Příklad: Výpočet diskriminantu} 
    \begin{minted}[linenos=true]{c} 
double diskriminant(double a, double b, double c) {
    double b_na_druhou = b * b;
    double diskr = b_na_druhou - (4 * a * c);
    return diskr;
}

int main() {
    double a = 2.26, b = 7.7, x = 3.4;
    double diskr = 0;
    double vysledek = diskriminant(b, a, x);
    printf("diskr: %g, vysledek: %g", diskr, vysledek);
    // diskr: 0, vysledek: -99.6124
    printf("b_na_druhou: %g", b_na_druhou); // CHYBA!
}
    \end{minted}
\end{frame}


\begin{frame}[t,fragile]\frametitle{Ukončením bloku zaniká proměnná} 
    \begin{itemizex}
        \item Pokud máme v libovolném bloku definovanou proměnnou, po skončení bloku proměnná zaniká
        \begin{minted}[linenos=true]{c} 
int main() {
    int a = 1;
    if (a == 1) {
        char znak = '!';
        printf("%c", znak);
        a = 3;
    }
    printf("%i", a); // V poradku
    printf("%c", znak); // CHYBA! Znak uz neexistuje
}
        \end{minted}
    \end{itemizex}
\end{frame}


\begin{frame}[t,fragile]\frametitle{Globální proměnná} 
    \begin{itemizex}
        \item Globální proměnnou definujeme mimo všechny bloky.
        \item Tj. mimo všechny funkce, tedy i mimo funkci \texttt{main}.
        \item Proměnná vzniká při spuštění programu a odstraňuje se při ukončení programu.
        \item Globální proměnné mohou být i nebezpečné.
    \end{itemizex}
\end{frame}


\begin{frame}[t,fragile]\frametitle{Příklad: Počet volání funkce} 
    \begin{itemize}
        \item Funkce, která si v globální proměnné \texttt{pocet\_volani} pamatuje, kolikrát jsme ji zavolali
        \begin{minted}[linenos=true]{c} 
int pocet_volani = 0;
void vypis_moudro(char moudro[]) {
    printf("Vypisuji moudro: %s\n", moudro);
    pocet_volani++;
}
int main() {
    vypis_moudro("Globalni promenne jsou uzitecne.");
    vypis_moudro("Ale ne vzdy.");
    vypis_moudro("Obcas jsou vylozene zle.");
    printf("%i", pocet_volani); // 3
    return 0;
}
        \end{minted}
    \end{itemize}
\end{frame}


\begin{frame}[t,fragile]\frametitle{Statická proměnná} 
    \begin{itemizex}
        \item Statická lokální proměnná je proměnná uvnitř bloku, která nezaniká s ukončením bloku, ale přežije do konce programu.
        \item Definujeme ji použitím klíčového slova \texttt{static}.
        \item K proměnné se ale dostaneme pouze v bloku, ve kterém proměnná existuje.
    \end{itemizex}
\end{frame}


\begin{frame}[t,fragile]\frametitle{Příklad: Počet volání funkce pomocí statické proměnné} 
    \begin{minted}[linenos=true]{c} 
int funkce() {
    static int pocet = 0;
    // udelej neco strasne uzitecneho
    pocet++;
    return pocet;
}

int main() {
    funkce();
    funkce();
    funkce();
    printf("pocet volani: %i", funkce());
}
    \end{minted}
\end{frame}




\end{document}