\documentclass{beamer}
\usetheme[pageofpages=z,
          bullet=circle,
          titleline=true,
          alternativetitlepage=true,
          titlepagelogo=uplogo,
          ]{UVT}

\usepackage[utf8x]{inputenc}
\usepackage{czech}
\usepackage{minted}
% \usepackage{url}


\newenvironment{itemizex}%
  {\large \begin{itemize}%
    \setlength{\itemsep}{8pt}%
    \setlength{\parskip}{8pt}}%
  {\end{itemize}}

\newenvironment{enumeratex}%
  {\large \begin{enumerate}%
    \setlength{\itemsep}{6pt}%
    \setlength{\parskip}{6pt}}%
  {\end{enumerate}}


\title{Základy programování 1 (KMI/ZP1 a KMI/UP1)}
\subtitle{Cvičení 5: Jednorozměrná pole}
\author{Lukáš Havrlant}
\date{17. října 2012}
\institute{Univerzita Palackého}

\begin{document}

\begin{frame}[t,plain]
\titlepage
\end{frame}


\begin{frame}[t,fragile]\frametitle{Co je to pole} 
  \begin{itemizex}
    \item Pole je datová struktura, která sdružuje daný konečný počet prvků (čísel) stejného datového typu.
    \item Jak bychom v Céčku uložili pět čísel? 
    \item Pět proměnných $\longrightarrow$ fuj!
    \item Použijeme pole: 
\begin{minted}[linenos=true]{c} 
int cisla[5] = {1, 3, 6, -5032, 42};
\end{minted}
  \end{itemizex}
\end{frame}


\begin{frame}[t,fragile]\frametitle{Definice pole} 
  \begin{itemizex}
    \item Obecná definice:
    \begin{minted}[linenos=true]{c} 
typ nazev[pocet_prvku];
    \end{minted}
    \item Příklad:
    \begin{minted}[linenos=true]{c} 
int prvocisla[7];
double odmocniny[10];

int suda_cisla[5] = {2, 4, 6, 8, 10};
int licha_cisla[] = {1, 3, 5};
double dane[] = {16.5, 18, 21.75};
    \end{minted}
  \end{itemizex}
\end{frame}



\begin{frame}[t,fragile]\frametitle{Přístup k prvkům pole} 
  \begin{itemize}
    \item Pomocí operátoru \texttt{[]}
    \item Pokud chceme získat/nastavit 4. prvek pole, použijeme kód:
    \begin{minted}[linenos=true]{c} 
int pole[] = {1, 2, 3, 4, 5, 6, 7};
int ctvrty_prvek = pole[4];
pole[2] = 0;
    \end{minted}
    \item Zrada: pole je číslováno od nuly, ne od jedničky!
    \begin{minted}[linenos=true]{c} 
printf("%d", pole[4]); // Vypise 5!
printf("%d", pole[0]); // Vypise 1!
printf("%d", pole[7]); // 7. prvek pole neexistuje
pole[2] = 0;
// Pole ma tvar:
// {1, 2, 0, 4, 5, 6, 7}
    \end{minted}
  \end{itemize}
\end{frame}


\begin{frame}[t,fragile]\frametitle{Inicializace prvků pole} 
  \begin{itemizex}
    \item Prvky pole nejsou inicializovány na žádnou výchozí hodnotu.
    \item Kód \texttt{int pole[5];} jen vytvoří v paměti místo pro pět integerů.
    \begin{minted}[linenos=true]{c} 
int cisla[5], i;
for (i = 0; i < 5; i++) 
    printf("%i", cisla[i]);
// Vypise napr. 0 0 1506385072 32767 1706455134
    \end{minted}
    \item Inicializujeme ručně:
    \begin{minted}[linenos=true]{c} 
for (i = 0; i < 5; i++) 
    cisla[i] = 0;
    \end{minted}
  \end{itemizex}
\end{frame}


\begin{frame}[t,fragile]\frametitle{Jak zjistíme délku pole?} 
    \begin{itemizex}
        \item Nijak.
    \end{itemizex}
\end{frame}


\begin{frame}[t,fragile]\frametitle{Jak si pamatovat délku pole?} 
    \begin{itemize}
        \item Bohužel v C89 (Visual Studio) nelze použít proměnnou
        \begin{minted}[linenos=true]{c} 
int delka = 42;
int pole[delka]; // Toto v C89 nefunguje!
        \end{minted}
        \item Použijte preprocesorovou direktivu \texttt{define}
        \item Konstantu pište velkými písmeny.
        \item Nebo použijte kompilátor podporující C99.
        \begin{minted}[linenos=true]{c} 
#define DELKA 10
int pole[DELKA], i;
for (i = 0; i < DELKA; i++) 
    pole[i] = i * i;
        \end{minted}
    \end{itemize}
\end{frame}


\begin{frame}[t,fragile]\frametitle{Textové řetězce} 
    \begin{itemize}
        \item V Céčku se textové řetězce, např. "Ahoj Svete!", realizují jako pole charů.
        \item Konec řetězce označuje znak '\textbackslash0', nula.
        \item Řetězec "ahoj" je rovný poli \texttt{\{'a', 'h', 'o', 'j', '\textbackslash0'\}}
        \begin{minted}[linenos=true]{c} 
char str[] = "ahoj";
int i;

for (i = 0; i <= 4; i++) 
    printf("%i ", str[i]);
// 97 104 111 106 0
        \end{minted}
    \end{itemize}
\end{frame}


\begin{frame}[t,fragile]\frametitle{Nulová zarážka v podmínce} 
    \begin{itemizex}
        \item Nulová zarážka má číselnou hodnotu 0
        \item Lze tak použít v podmínce cyklu pro zjištění konce řetězce:
        \begin{minted}[linenos=true]{c} 
char str[] = "ahoj";
char znak;
int i = 0;

while (znak = str[i]) {
    printf("%c ", znak);
    i++;
}
// a h o j 
        \end{minted}
    \end{itemizex}
\end{frame}



\begin{frame}[t,fragile]\frametitle{Jak vypsat řetězec v printf} 
    \begin{itemizex}
        \item Funkce \texttt{printf} umí s řetězci pracovat.
        \item Stačí použít modifikátor \texttt{\%s}:
        \begin{minted}[linenos=true]{c} 
char str[] = "ahoj";
printf("%s", str);
// ahoj
        \end{minted}
        \item Je to stejné jako
        \begin{minted}[linenos=true]{c} 
while (znak = str[i]) {
    printf("%c", znak);
    i++;
}
        \end{minted}
    \end{itemizex}
\end{frame}


\begin{frame}[t,fragile]\frametitle{Nulová zarážka fakt ukončuje řetězec} 
    \begin{minted}[linenos=true]{c} 
char pozdrav[] = "Dobry vecer!";
printf("'%s'\n", pozdrav); // 'Dobry vecer!'
pozdrav[5] = '\0';
printf("'%s'\n", pozdrav); // 'Dobry'
    \end{minted}
    \begin{itemizex}
        \item Řetězec má nyní tvar:
        \item \texttt{\{'D', 'o', 'b', 'r', 'y', '\textbackslash0', 'v', 'e', 'c', 'e', 'r', '\textbackslash0'\}}
    \end{itemizex}
    \begin{minted}[linenos=true]{c} 
printf("'%c'", pozdrav[10]); // Vypise: 'r'
pozdrav[5] = '_';
printf("'%s'\n", pozdrav); // 'Dobry_vecer!'
    \end{minted}
\end{frame}


\begin{frame}[t,fragile]\frametitle{Řetězec bez nulové zarážky} 
    \begin{itemizex}
        \item Pokud zrušíme nulovou zařážku, bude to vypisovat nesmysly.
        \begin{minted}[linenos=true]{c} 
char pozdrav[] = "Dobry vecer!";
pozdrav[12] = '+';
printf("'%s'\n", pozdrav); // 'Dobry vecer!+??Q?'
        \end{minted}
        \item Pokud u pole charů specifikujete velikost, musí být pole o jedna větší než řetězec.
        \begin{minted}[linenos=true]{c} 
// V poli bude chybet nulova zarazka
char spatny_pozdrav[4] = "ahoj"; 
char spravny_pozdrav[5] = "ahoj";
        \end{minted}
    \end{itemizex}
\end{frame}


\begin{frame}[t,fragile]\frametitle{Jeden hodně stručný while} 
    \begin{itemizex}
        \item Je k něčemu dobrý tento \texttt{while}?
        \begin{minted}[linenos=true]{c} 
char str[] = "slovo";
int i = -1;
while (str[++i]);
    \end{minted}
    \end{itemizex}
\end{frame}


\begin{frame}[t,fragile]\frametitle{Tady začíná peklo} 
    \begin{itemizex}
        \begin{minted}[linenos=true]{c} 
int suda[5] = {2, 4, 6, 8, 10};
int licha[5] = {1, 3, 5, 7, 9};
suda[5] = 42;
        \end{minted}
        \item Pole \texttt{suda} nemá pátý index $\longrightarrow$ co se stane?
        \item Lepší možnost: program spadne (Segmentation fault), protože přistupuje do paměti, kterou nevlastní. 
        \item Horší možnost: \uv{Někam} do paměti se uloží číslo 42.
        \item !!!\underline{\textbf{NĚKAM}}!!! se uloží $\longrightarrow$ ZLO!!!
        % \item (Přesněji -- my víme kam, akorát nevíme, co tam je.)
    \end{itemizex}
\end{frame}


\begin{frame}[t,fragile]\frametitle{Vizualizace pekla} 
\begin{minted}[linenos=true]{c} 
----------------------------
...|2|4|6|8|10|1|3|5|7|9|...
----------------------------
    ^suda      ^licha
\end{minted}
    \begin{itemizex}
        \item Může se stát, že se do pole \texttt{licha} uloží 42 na první pozici.
        \begin{minted}[linenos=true]{c} 
suda[5] = 42;
printf("%i", licha[0]); // Muze vypsat 42
        \end{minted}
    \end{itemizex}
\end{frame}


\begin{frame}[t,fragile]\frametitle{Ještě jeden příklad}
\begin{minted}[linenos=true]{c} 
int pole[3] = {4, 8, 10};
char text[] = "Tohle je hodne zle.";
pole[3] = 0;
printf("'%s'", text); // Muze to vypsat ''
\end{minted}
    \begin{itemizex}
        \item Opět se nula může uložit na místo \texttt{text[0]}.
        \item Jenže nula ukončuje řetězec: \texttt{0 == '\textbackslash0'}
        \item V proměnné \texttt{text} tak máme prázdný řetězec.
    \end{itemizex}
\end{frame}


\begin{frame}[t,fragile]{Úlohy}
\begin{center}
\vskip 1cm
{\Large Seznam úloh je na \url{http://KMIup1.jdem.cz}}
\vskip 2cm
\url{http://tux.inf.upol.cz/~havrlant/}
\end{center}
\end{frame}


\end{document}