\documentclass{beamer}
\usetheme[pageofpages=z,
          bullet=circle,
          titleline=true,
          alternativetitlepage=true,
          titlepagelogo=uplogo,
          ]{UVT}

\usepackage[utf8x]{inputenc}
\usepackage{czech}
\usepackage{minted}
% \usepackage{url}


\newenvironment{itemizex}%
  {\large \begin{itemize}%
    \setlength{\itemsep}{8pt}%
    \setlength{\parskip}{8pt}}%
  {\end{itemize}}

\newenvironment{enumeratex}%
  {\large \begin{enumerate}%
    \setlength{\itemsep}{6pt}%
    \setlength{\parskip}{6pt}}%
  {\end{enumerate}}


\title{Základy programování 1 (KMI/ZP1 a KMI/UP1)}
\subtitle{Poznámky k úkolům z cvičení 5}
\author{Lukáš Havrlant}
\date{24. října 2012}
\institute{Univerzita Palackého}

\begin{document}

\begin{frame}[t,plain]
\titlepage
\end{frame}


\begin{frame}[t,fragile]\frametitle{Nepoužívejte číselné konstanty} 
\begin{minted}[linenos=true]{c} 
int sito[100];
for(i = 2; i < 100; i++)
    sito[i] = 1;

for(i = 2; i < 100; i++){
    if(sito[i] == 0) continue;
    for(j = i * 2; j < 100; j += i)
        sito[j] = 0;
}    

for(i = 2; i < 100; i++) {
    if(sito[i] == 1)
        printf("%d ", i);
}
\end{minted}
\end{frame}


\begin{frame}[t,fragile]\frametitle{Konstantu si někam uložte} 
\begin{minted}[linenos=true]{c} 
#define POCET_PRVOCISEL 100
int sito[100];
for(i = 2; i < POCET_PRVOCISEL; i++)
    sito[i] = 1;

for(i = 2; i < POCET_PRVOCISEL; i++){
    if(sito[i] == 0) continue;
    for(j = i * 2; j < POCET_PRVOCISEL; j += i)
        sito[j] = 0;
}    

for(i = 2; i < POCET_PRVOCISEL; i++) {
    if(sito[i] == 1)
        printf("%d ", i);
}
\end{minted}
\end{frame}


\begin{frame}[t,fragile]\frametitle{Nepoužívejte zbytečně continue} 
\begin{minted}[linenos=true]{c} 
for(i = 2; i < POCET_PRVOCISEL; i++){
    if(sito[i] == 0) continue;
    for(j = i * 2; j < POCET_PRVOCISEL; j += i)
        sito[j] = 0;
}

// Vetsinou byva prehlednsi tato varianta:
for(i = 2; i < POCET_PRVOCISEL; i++){
    if(sito[i] > 0) {
        for(j = i * 2; j < POCET_PRVOCISEL; j += i)
            sito[j] = 0;
    }
}  
\end{minted}
\end{frame}


\begin{frame}[t,fragile]\frametitle{Používejte různé konstanty} 
\begin{minted}[linenos=true]{c} 
// ukol 1
#define DELKA 10

// ukol 3
#define DELKA 100

// Konstanta by mela byt v ramci programu unikatni
// Takze lepe nejak takto. Idealne, aby byly definice hned pod includy
#include <stdio.h>
#define DELKA_SUMA 10
#define POCET_PRVOCISEL 100
// ukol 1
// ukol 3

\end{minted}
\end{frame}


\begin{frame}[t,fragile]\frametitle{Kód nemá být namačkaný na sebe} 
\begin{minted}[linenos=true]{c} 
void soucet_prefixu()  {
    int i, soucet, cisla[5] = {1,2,5,7,9};
    for (i = 0; i < 5; i++)
        printf("%i ",cisla[i]);
    for (soucet = 0, i = 0; i < 5; i++) {
        soucet = soucet + cisla[i];
        printf("%i ", soucet);
    }
}
void cetnost_znaku() {
    char text[] = "aoddr ff223ca ambte dfghn!";
    int cetnost[256];
    int i, j;
    for (i = 0; i < 256; i++)... 
\end{minted}
\end{frame}


\begin{frame}[t,fragile]\frametitle{Používejte prázdné řádky} 
\begin{minted}[linenos=true]{c} 
void soucet_prefixu()  {
    int i, soucet, cisla[5] = {1,2,5,7,9};

    for (i = 0; i < 5; i++)
        printf("%i ",cisla[i]);

    for (soucet = 0, i = 0; i < 5; i++) {
        soucet = soucet + cisla[i];
        printf("%i ", soucet);
    }
}

void cetnost_znaku() {
    char text[] = "aoddr ff223ca ambte dfghn!";
\end{minted}
\end{frame}


\begin{frame}[t,fragile]\frametitle{Ale přidávejte je s citem!} 
\begin{minted}[linenos=true]{c} 
for(x=0;x<100;x++)
            {
                if(poleprvku[x] == 0)

                    continue;




                printf("%d\t", poleprvku[x]);

                for(z=2 * x;z<100;z=z+x)
                {

                  poleprvku[z]=0;


                }


            }
\end{minted}
\end{frame}


\begin{frame}[t,fragile]\frametitle{Nepoužívejte vícekrát include} 
\begin{minted}[linenos=true]{c} 
// ukol 1
#include <stdio.h>

// ukol 2
#include <stdio.h>

// ukol 3
#include <stdio.h>

// Spravne jen jeden include na zacatku souboru
\end{minted}
\end{frame}


\begin{frame}[t,fragile]\frametitle{Jak projít řetězec} 
\begin{minted}[linenos=true]{c} 
int main() {
    char znak;
    char retezec[] = "ahoj";
    int delka = -1, i;
    while(retezec[++delka]);

    for (i = 0; i < delka; i++)
        printf("%c", retezec[i]);

    // lepe:
    i = 0;
    while(znak = retezec[i]) {
        printf("%c", retezec[i]);
        i++;
    }
}
\end{minted}
\end{frame}

\end{document}