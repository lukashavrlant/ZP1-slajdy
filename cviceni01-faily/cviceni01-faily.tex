\documentclass{beamer}
\usetheme[pageofpages=z,
          bullet=circle,
          titleline=true,
          alternativetitlepage=true,
          titlepagelogo=uplogo,
          ]{UVT}

\usepackage[utf8x]{inputenc}
\usepackage{czech}
\usepackage{minted}
% \usepackage{url}


\newenvironment{itemizex}%
  {\large \begin{itemize}%
    \setlength{\itemsep}{8pt}%
    \setlength{\parskip}{8pt}}%
  {\end{itemize}}

\newenvironment{enumeratex}%
  {\large \begin{enumerate}%
    \setlength{\itemsep}{6pt}%
    \setlength{\parskip}{6pt}}%
  {\end{enumerate}}


\title{Úvod do programování 1 (KMI/ZP1 a KMI/UP1)}
\subtitle{Poznámky ke cvičení 1}
\author{Lukáš Havrlant}
\date{19. září 2012}
\institute{Univerzita Palackého}

\begin{document}

\begin{frame}[t,plain]
\titlepage
\end{frame}

\begin{frame}[t,fragile]\frametitle{Visual Studio} 
  \begin{itemizex}
    \item Při vytváření projektu nezapomeňte na volbu \textbf{Empty Project}.
    \item V jednom projektu může být jen jedna \texttt{main} funkce. 
    \item Pro různé úlohy buď vytvářet nové projekty\dots
    \item \dots nebo všechno cpát do jedné \texttt{main} funkce.
    \item Nepoužívejte nikde diakritiku: \texttt{Ahoj světě!} $\longrightarrow$ \texttt{Ahoj svete!}
  \end{itemizex}
\end{frame}



\begin{frame}[t,fragile]\frametitle{Standardy jazyka C} 
  \begin{itemizex}
    \item Existuje několik standardů jazyka C. 
    \item ANSI C, někdy také C89 a C90 z roku 1989, resp. 1990.
    \item C99 z roku 1999.
    \item C11 z roku 2011. 
  \end{itemizex}
\end{frame}


\begin{frame}[t,fragile]\frametitle{Významné novinky v C99} 
  \begin{itemize}
    \item Podpora pro jednořádkové komentáře: \texttt{// komentar}
    \item \texttt{long long int}
    \item Proměnné nemusí být deklarovány na začátku bloku.
    \begin{minted}[linenos=true]{c} 
int main() {
    // VS zvladne tyto komentare
    printf("Hello, World!\n");

    /* Zvladne i typ long long, 
    ale musi byt deklarovan jeste pred printf */
    long long cislo = 10; 
    return 0;
}
    \end{minted}
    \item VS podporuje C89/C90 a umí některé věci z C99.
  \end{itemize}
\end{frame}


\begin{frame}[t,fragile]\frametitle{Short int} 
  \begin{itemizex}
    \item Typ \texttt{short} se používá pro čísla malého rozsahu.
    \item Např. počet studentů v ročníku uložíme do \texttt{short}.
    \item Rozsah \texttt{short}u: od -32768 do 32767.
    \item Na uložení malých čísel jako $0.000023$ použijeme typ \texttt{double}!
    \item Nejjednodušší je používat \texttt{char} pro znaky, \texttt{int} pro celá čísla a \texttt{double} pro racionální čísla.
    \item V \texttt{printf} pak \texttt{\%c} pro znaky, \texttt{\%d} (nebo \texttt{\%i}) pro celá čísla a \texttt{\%g} pro racionální čísla.
  \end{itemizex}
\end{frame}



\end{document}