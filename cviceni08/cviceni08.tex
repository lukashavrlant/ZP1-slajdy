\documentclass{beamer}
\usetheme[pageofpages=z,
          bullet=circle,
          titleline=true,
          alternativetitlepage=true,
          titlepagelogo=uplogo,
          ]{UVT}

\usepackage[utf8x]{inputenc}
\usepackage{czech}
\usepackage{minted}
\usepackage{color}
\usepackage{amssymb}
\usepackage{array}
\usepackage{graphicx}
\newcolumntype{C}[1]{>{\centering\let\newline\\\arraybackslash\hspace{0pt}}m{#1}}

% \usepackage{url}


\newenvironment{itemizex}%
  {\large \begin{itemize}%
    \setlength{\itemsep}{8pt}%
    \setlength{\parskip}{8pt}}%
  {\end{itemize}}

\newenvironment{enumeratex}%
  {\large \begin{enumerate}%
    \setlength{\itemsep}{6pt}%
    \setlength{\parskip}{6pt}}%
  {\end{enumerate}}


\title{Základy programování 1 (KMI/ZP1 a KMI/UP1)}
\subtitle{Cvičení 8: Strukturované datové typy}
\author{Lukáš Havrlant}
\date{7. října 2012}
\institute{Univerzita Palackého}

\begin{document}

\begin{frame}[t,plain]
\titlepage
\end{frame}


\begin{frame}[t,fragile]\frametitle{Poznámka z úkolům} 
\begin{center}
\vskip 1cm
{\LARGE NIKDY NEPOUŽÍVEJTE\\
\vskip 5mm 
GLOBÁLNÍ PROMĚNNÉ!!!
}
\\
\vskip 1cm
{\small (Leda byste fakt museli.)}
\end{center}
\end{frame}


\begin{frame}[t,fragile]\frametitle{Poznámka k úkolům} 
\begin{minted}[linenos=true]{c} 
int i, soucet = 0, temp;
void obrat_pole(int pole[], int delka) {
    for (i = 0; i < delka / 2; i++) {
        temp = pole[i];
        pole[i] = pole[delka - i - 1];
        pole[delka - i - 1] = temp;
    }
}
int soucet_cifer(int cislo) {
    while (cislo > 0) {
        soucet += cislo % 10;
        cislo /= 10;
    }
    return soucet;
}
\end{minted}
\end{frame}

\begin{frame}[t,fragile]\frametitle{George Boole} 
    \begin{itemizex}
        \item Britský matematik, 1815--1864
        \item Jeden ze zakladatelů informatiky
        \item Spolupracoval s A. De Morganem: $\neg(p\vee q) = (\neg p) \wedge (\neg q)$
        \item Je po něm pojmenována Booleova algebra
        \item \dots a kráter na Měsíci
    \end{itemizex}
\end{frame}


\begin{frame}[t,fragile]\frametitle{Motivace} 
    \begin{itemizex}
        \item Mějme funkci, která zjišťuje, zda je v poli nula:
        \begin{minted}[linenos=true]{c} 
int obsahuje_nulu(int cisla[], int delka) {
    int i;
    for (i = 0; i < delka; i++)
        if (cisla[i] == 0) 
            return 1;
    return 0;
}
        \end{minted}
        \item Jak zařídit, abychom poznali, že funkce vrací jen pravda/nepravda?
        \item Jak zařídit, abychom mohli psát TRUE/FALSE místo 0/1?
    \end{itemizex}
\end{frame}


\begin{frame}[t,fragile]\frametitle{Čeho chceme dosáhnout} 
    \begin{itemizex}
        \item Boolean, Bool -- název typu pro pravdu/nepravdu. 
    \end{itemizex}
        \begin{minted}[linenos=true]{c} 
Bool obsahuje_nulu(int cisla[], int delka) {
    int i;
    for (i = 0; i < delka; i++)
        if (cisla[i] == 0) 
            return TRUE;
    return FALSE;
}
        \end{minted}
\end{frame}

\begin{frame}[t,fragile]\frametitle{Řešení pomocí konstant} 
\begin{minted}[linenos=true]{c} 
#define Bool int
#define TRUE 1
#define FALSE 0

Bool obsahuje_nulu(int cisla[], int delka) {
    int i;
    for (i = 0; i < delka; i++)
        if (cisla[i] == 0) 
            return TRUE;
    return FALSE;
}
\end{minted}
\end{frame}


\begin{frame}[t,fragile]\frametitle{Konstrukce \texttt{typedef}} 
    \begin{itemizex}
        \item Pomocí příkazu \texttt{typedef} můžeme definovat nová vlastní jména existujícím typům.
        \begin{minted}[linenos=true]{c} 
typedef zapis_ve_tvaru_deklarace;
        \end{minted}
        \item Použití:
        \begin{minted}[linenos=true]{c} 
typedef int Bool;
        \end{minted}
        \item Typ \texttt{Bool} se stane aliasem pro typ \texttt{int}
    \end{itemizex}
\end{frame}


\begin{frame}[t,fragile]\frametitle{Řešení pomocí \texttt{typedef}} 
\begin{minted}[linenos=true]{c} 
typedef int Bool;
#define TRUE 1
#define FALSE 0

Bool obsahuje_nulu(int cisla[], int delka) {
    int i;
    for (i = 0; i < delka; i++)
        if (cisla[i] == 0) 
            return TRUE;
    return FALSE;
}
\end{minted}
\end{frame}


\begin{frame}[t,fragile]\frametitle{Další příklady použití konstrukce \texttt{typedef}} 
\begin{minted}[linenos=true]{c} 
typedef int cele_cislo; 
typedef int male_pole[10]; 
typedef char retezec[];

int main() {
    int postaru = 47;
    cele_cislo novy_zpusob = 47;
    male_pole cisla;
    retezec film = "Braindead - Zivi mrtvi";
    return 0;     
}
\end{minted}
\end{frame}



\begin{frame}[t,fragile]\frametitle{Použití výčtového typu} 
    \begin{itemizex}
        \item Výčtový typ umožňuje definovat nový typ a konstanty zároveň.
        \item Výčtový typ definujeme pomocí konstrukce \texttt{enum}.
    \end{itemizex}
\begin{minted}[linenos=true]{c} 
typedef enum{TRUE = 1, FALSE = 0} Bool;

Bool obsahuje_nulu(int cisla[], int delka) {
    int i;
    for (i = 0; i < delka; i++)
        if (cisla[i] == 0) 
            return TRUE;
    return FALSE;
}
\end{minted}
\end{frame}


\begin{frame}[t,fragile]\frametitle{Příklad druhý, dny v týdnu} 
\begin{minted}[linenos=true]{c} 
typedef enum{PO, UT, ST, CT, PA, SO, NE} Den;
// typedef enum{PO = 0, UT = 1, ST = 2, 
// CT = 3, PA = 4, SO = 5, NE = 6} Den;

int main() {
    Den statnice;
    Den cviceniZP1 = ST;
    Den dnes = SO;
    if (dnes == PO) 
        rychle_dodelat_ukoly();
    printf("Den: %i\n", UT); // Den: 1
    return 0;     
}
\end{minted}
\end{frame}


\begin{frame}[t,fragile]\frametitle{Poznámka z úkolům: opakování} 
\begin{center}
\vskip 1cm
{\LARGE NIKDY NEPOUŽÍVEJTE\\
\vskip 5mm 
GLOBÁLNÍ PROMĚNNÉ!!!
}
\\
\vskip 1cm
{\small (Leda byste fakt museli.)}
\end{center}
\end{frame}


\begin{frame}[t,fragile]\frametitle{Struktury: motivace} 
    \begin{itemizex}
        \item Jak reprezentovat \uv{datum}? Např. 18. 2. 2012.
        \item Jak předat datum jako parametr funkce a jak vrátit datum?
\begin{minted}[linenos=true]{c} 
typedef enum{PO, UT, ST, CT, PA, SO, NE} Den;
Den den(short rok, char mesic, char den) {
    if (...) 
        return ST; ...
}

// vstupem je n-ty den roku 2012
??? datum(short den) {
    return ???
}
\end{minted}
    \end{itemizex}
\end{frame}


\begin{frame}[t,fragile]\frametitle{Struktura} 
    \begin{itemizex}
        \item Pole nemůžeme použít kvůli různým typům: \texttt{char} $\times$ \texttt{short}.
        \item Použijeme strukturu: 
        \begin{minted}[linenos=true]{c} 
typedef struct {
   char Den, Mesic; 
   short Rok;
} Datum;
        \end{minted}
        \item Struktura je \uv{něco jako pole} -- jednotlivé buňky ale nemáme očíslované, ale pojmenované.
        \item Každý člen může mít jiný typ.
    \end{itemizex}
\end{frame}

\begin{frame}[t,fragile]\frametitle{Rozdíl mezi polem a strukturou} 
\begin{minted}[linenos=true]{c} 
Datum dnes;                   short cisla[3];
dnes.Den = 27;                cisla[0] = 27;
dnes.Mesic = 11;              cisla[1] = 11;
dnes.Rok = 2012;              cisla[2] = 2012;
\end{minted}
\vskip 5mm
\begin{itemizex}
    \item V poli máme třikrát \texttt{short}, ve struktuře dvakrát \texttt{char} a jednou \texttt{short}!
\end{itemizex}

\begin{center}
    \begin{tabular}{c|c}
        {\large dnes:}&{\large cisla:}\\\hline
        \begin{tabular}{C{10mm}C{10mm}C{20mm}}
            Den&Mesic&Rok\\
            $\downarrow$&$\downarrow$&$\downarrow$
        \end{tabular}
        &
        \begin{tabular}{C{10mm}C{10mm}C{10mm}}
            0&1&2\\
            $\downarrow$&$\downarrow$&$\downarrow$
        \end{tabular}
        \\
        \begin{tabular}{|C{10mm}|C{10mm}|C{20mm}|}
            \hline 27&11  &2012 \\\hline
        \end{tabular}
        &
        \begin{tabular}{|C{10mm}|C{10mm}|C{10mm}|}
            \hline 27&11  &2012 \\\hline
        \end{tabular}
        \\
        \begin{tabular}{C{10mm}C{10mm}C{20mm}}
            {\small \texttt{char}}&{\small \texttt{char}}&{\small \texttt{short}}
        \end{tabular}
        &
        \begin{tabular}{C{10mm}C{10mm}C{10mm}}
            {\small \texttt{short}}&{\small \texttt{short}}&{\small \texttt{short}}
        \end{tabular}
    \end{tabular}
    
\end{center}
\end{frame}


\begin{frame}[t,fragile]\frametitle{Jak pracovat se strukturami} 
\begin{minted}[linenos=true]{c} 
typedef struct {
   char Den, Mesic; 
   short Rok;
} Datum;

int main() {
    Datum narozen = {24, 3, 1972}; // zkraceny zapis
    narozen.Rok = 1986;
    narozen.Mesic = narozen.Mesic + 1;
    narozen.Mesic++;
    printf("Den narozeni: %i", narozen.Den);
}
\end{minted}
\end{frame}


\begin{frame}[t,fragile]\frametitle{Zlomky} 
\begin{minted}[linenos=true]{c} 
typedef struct {
    int citatel, jmenovatel;
} Zlomek;
void vytiskni_zlomek(Zlomek zlomek) {
    printf("%i/%i", zlomek.citatel, zlomek.jmenovatel);
}
int main() {
    Zlomek z1 = {2, 5}, z2 = {3, 7}, z3;
    z3.citatel = 0;
    z3.jmenovatel = 45;
    vytiskni_zlomek(z1); // Vytiskne: 2/5
    vytiskni_zlomek(z2); // Vytiskne: 3/7
    vytiskni_zlomek(z3); // Vytiskne: 0/45
}
\end{minted}
\end{frame}


\begin{frame}[t,fragile]\frametitle{Součin zlomků} 
\begin{minted}[linenos=true]{c} 
Zlomek soucin_zlomku(Zlomek z1, Zlomek z2) {
    Zlomek novy_zlomek;
    novy_zlomek.citatel = z1.citatel * z2.citatel;
    novy_zlomek.jmenovatel = z1.jmenovatel * z2.jmenovatel;
    return novy_zlomek;
}

int main() {
    Zlomek z1 = {2, 5}, z2 = {3, 7}, z3;
    z3.citatel = 0;
    z3.jmenovatel = 45;
    vytiskni_zlomek(soucin_zlomku(z1, z2)); // 6/35
    vytiskni_zlomek(soucin_zlomku(z1, z3)); // 0/225
    vytiskni_zlomek(soucin_zlomku(z2, z3)); // 0/315
}
\end{minted}
\end{frame}



\begin{frame}[t,fragile]\frametitle{Struktura může obsahovat strukturu} 
\begin{minted}[linenos=true]{c} 
typedef struct {
    Zlomek citatel_sl, jmenovatel_sl;
} SlozenyZlomek;

int main() {
    Zlomek z1 = {2, 5}, z2 = {3, 7};
    SlozenyZlomek slozeny = {z1, z2};
    vytiskni_zlomek(slozeny.citatel_sl); // Vytiskne: 2/5
    vytiskni_zlomek(slozeny.jmenovatel_sl); // Vytiskne: 3/7
    printf("%i\n", slozeny.citatel_sl.jmenovatel); // 5
    printf("%i\n", slozeny.jmenovatel_sl.citatel); // 3
    slozeny.jmenovatel_sl.jmenovatel = 10;
    vytiskni_zlomek(slozeny.jmenovatel_sl); // Vytiskne: 3/10
}
\end{minted}
\end{frame}


\begin{frame}[t,fragile]\frametitle{Diagram struktury ve struktuře} 
\begin{figure}[htb]
\centering
\includegraphics[width=0.85\textwidth]{slzlomek.pdf}
\end{figure}
\end{frame}


\begin{frame}[t,fragile]{Úlohy}
\begin{center}
\vskip 1cm
{\Large Seznam úloh je na \url{http://KMIup1.jdem.cz}}
\vskip 2cm
\url{http://tux.inf.upol.cz/~havrlant/}
\end{center}
\end{frame}


\end{document}