\documentclass{beamer}
\usetheme[pageofpages=z,
          bullet=circle,
          titleline=true,
          alternativetitlepage=true,
          titlepagelogo=uplogo,
          ]{UVT}

\usepackage[utf8x]{inputenc}
\usepackage{czech}
\usepackage{minted}
% \usepackage{url}


\newenvironment{itemizex}%
  {\large \begin{itemize}%
    \setlength{\itemsep}{8pt}%
    \setlength{\parskip}{8pt}}%
  {\end{itemize}}

\newenvironment{enumeratex}%
  {\large \begin{enumerate}%
    \setlength{\itemsep}{6pt}%
    \setlength{\parskip}{6pt}}%
  {\end{enumerate}}


\title{Základy programování 1 (KMI/ZP1 a KMI/UP1)}
\subtitle{Cvičení 4: Cykly}
\author{Lukáš Havrlant}
\date{10. října 2012}
\institute{Univerzita Palackého}

\begin{document}

\begin{frame}[t,plain]
\titlepage
\end{frame}


\begin{frame}[t,fragile]\frametitle{K minulému cvičení\dots} 
  \begin{itemizex}
    \item Jak zjistit, jestli je znak malé číslo?
    \item \texttt{'a'} odpovídá číslu 97, \texttt{'z'} pak 122.
    \item Můžeme tak napsat \texttt{(znak >= 97) \&\& (znak <= 122)}
    \item Je to funkční, ale fuj!
    \item Správně: \texttt{(znak >= 'a') \&\& (znak <= 'z')}
  \end{itemizex}
\end{frame}


\begin{frame}[t,fragile]\frametitle{Cykly: Motivace} 
  \begin{itemizex}
    \item Úkol: vypište na obrazovku prvních 5 přirozených čísel. 
    \begin{minted}[linenos=true]{c} 
int main() {
    printf("1 ");
    printf("2 ");
    printf("3 ");
    printf("4 ");
    printf("5 ");
}
// 1 2 3 4 5
    \end{minted}
    \item Fuj!
  \end{itemizex}
\end{frame}


\begin{frame}[t,fragile]\frametitle{Lepší motivace} 
  \begin{itemize}
    \item Úkol: vypište na obrazovku prvních $n$ přirozených čísel.
    \begin{minted}[linenos=true]{c} 
int main() {
    int n = 3;
    if (n >= 1) printf("1 ");
    if (n >= 2) printf("2 ");
    if (n >= 3) printf("3 ");
    if (n >= 4) printf("4 ");
    if (n >= 5) printf("5 ");
    ...
}
// 1 2 3
    \end{minted}
    \item Fuj!
  \end{itemize}
\end{frame}


\begin{frame}[t,fragile]\frametitle{Cyklus while} 
  \begin{itemizex}
    \item Slouží k opakování části kódu, dokud je splněná nějaká podmínka
    \begin{minted}[linenos=true]{c} 
int i = 1;

while(i <= 3) {
    printf("%d ", i);
    i++;
}

// 1 2 3
    \end{minted}
  \end{itemizex}
\end{frame}

\begin{frame}[t,fragile]\frametitle{Jak bude probíhat výpočet} 
  \begin{minted}[linenos=true]{c} 
int i = 1;             
if (1 <= 3) {                 if (4 <= 3) {
    printf("%d ", 1);            printf("%d ", 4);
    i++; // i = 2                 i++;
}                             }
if (2 <= 3) {
    printf("%d ", 2);
    i++; // i = 3
}
if (3 <= 3) {
    printf("%d ", 2);
    i++; // i = 4
}
  \end{minted}
\end{frame}


\begin{frame}[t,fragile]\frametitle{Příklad} 
  \begin{itemize}
    \item Příklad:
    \begin{minted}[linenos=true]{c} 
int i = 10;
while(i) {
  printf("%i ", i);
  i--;
}
// 10 9 8 7 6 5 4 3 2 1 
    \end{minted}
  \item Totéž jako:
  \begin{minted}[linenos=true]{c} 
int i = 10;
while(i != 0) {
  printf("%i ", i);
  i--;
}
    \end{minted}
  \end{itemize}
\end{frame}


\begin{frame}[t,fragile]\frametitle{Zacyklení} 
  \begin{itemizex}
    \item Pokud zapíšeme blbě podmínku, můžeme získat nekonečný cyklus:
    \begin{minted}[linenos=true]{c} 
int i = 0, n = 10;
while (i < n) {
    printf("%i ", i);
}
    \end{minted}
    \item Nejspíš jsme zapomněli na \texttt{i++}
  \end{itemizex}
\end{frame}


\begin{frame}[t,fragile]\frametitle{Cyklus nemusí proběhnout ani jednou} 
  \begin{itemizex}
    \item Pokud není ani poprvé splněna podmínka, příkazy v těle cyklu \texttt{while} se neprovedou!
    \begin{minted}[linenos=true]{c} 
int i = 10, n = 10;

printf("Jsem pred cyklem. ");
while (i < n) {
    printf("Jsem v cyklu. ");
    i++;
}
printf("Jsem za cyklem.");
// Jsem pred cyklem. Jsem za cyklem.
    \end{minted}
  \end{itemizex}
\end{frame}


\begin{frame}[t,fragile]\frametitle{Cyklus \texttt{do-while}} 
  \begin{itemizex}
    \item Podobný cyklu \texttt{while}, ale příkazy v těle cyklu se vždy provedou alespoň jednou.
    \begin{minted}[linenos=true]{c} 
int i = 10, n = 10;

printf("Jsem pred cyklem. ");
do {
    printf("Jsem v cyklu. ");
    i++;
} while (i < n);
printf("Jsem za cyklem.");
// Jsem pred cyklem. Jsem v cyklu. Jsem za cyklem.
    \end{minted}
  \end{itemizex}
\end{frame}


\begin{frame}[t,fragile]\frametitle{Cyklus for} 
  \begin{itemizex}
    \item Klasické schéma pro \texttt{while}:
    \begin{minted}[linenos=true]{c} 
int i = 0;
while (i < 10) {
    // udelej neco...
    i++;
}
    \end{minted}
    \item Lze přehledněji napsat cyklem \texttt{for}:
    \begin{minted}[linenos=true]{c} 
int i;
for (i = 0; i < 10; i++) {
    // udelej neco...
}
    \end{minted}
  \end{itemizex}
\end{frame}


\begin{frame}[t,fragile]\frametitle{Syntaxe cyklu \texttt{for}} 
  \begin{itemizex}
    \item Obecná syntaxe cyklu \texttt{for}:
    \begin{minted}[linenos=true]{c} 
for (inicialize; podminka; iterace)
    prikaz
    \end{minted}
    \item \texttt{inicializace}: zde inicializujeme proměnné, více proměnných oddělujeme čárkou, např: \texttt{i = 10, j = 5}
    \item \texttt{podminka}: ukončovací podmínka cyklu, např. \texttt{i < 10}
    \item \texttt{iterace}: příkazy, které se provedou na konci cyklu, typicky změna proměnné, přes kterou iterujeme. Více příkazů opět oddělujeme čárou, např. \texttt{i++, j--}
  \end{itemizex}
\end{frame}


\begin{frame}[t,fragile]\frametitle{Cyklus \texttt{for} lze zapsat jako cyklus \texttt{while}} 
\begin{minted}[linenos=true]{c} 
for (inicialize; podminka; iterace)
    prikaz

inicialize;
while(podminka) {
    prikaz;
    iterace;
}
\end{minted}
\end{frame}


\begin{frame}[t,fragile]\frametitle{Příklady na \texttt{for}}
  \begin{itemize}
    \item Vypiš prvních $n$ přirozených čísel.
\begin{minted}[linenos=true]{c} 
int i, n = 5;
for (i = 1; i <= n; i++) {
    printf("%i ", i);
}
// 1 2 3 4 5
\end{minted}
  \item Vypiš prvních $n$ sudých čísel.
\begin{minted}[linenos=true]{c} 
int i, n = 5;
for (i = 2; i <= n * 2; i += 2) {
    printf("%i ", i);
}
// 2 4 6 8 10
\end{minted}
  \end{itemize}
\end{frame}



\begin{frame}[t,fragile]\frametitle{Další příklad na \texttt{for}} 
\begin{itemizex}
  \item Vypiš prvních $n$ přirozených čísel a u každého zobraz, zda je číslo liché nebo sudé.
\begin{minted}[linenos=true]{c} 
int i, n = 5;
for (i = 1; i <= n; i++) {
    if (i % 2 == 0)
        printf("%i je sude cislo\n", i);
    else
        printf("%i je liche cislo\n", i);
}
\end{minted}
\end{itemizex}
\end{frame}


\begin{frame}[t,fragile]\frametitle{Cykly můžeme vnořovat} 
\begin{minted}[linenos=true]{c} 
int i, j, n = 3;
for (i = 1; i <= n; i++) {
    for (j = 1; j <= n; j++) {
        printf("i=%i, j=%i | ", i, j);
    }
    printf("\n");
}
// i=1, j=1 | i=1, j=2 | i=1, j=3 | 
// i=2, j=1 | i=2, j=2 | i=2, j=3 | 
// i=3, j=1 | i=3, j=2 | i=3, j=3 |
\end{minted}
\end{frame}


\begin{frame}[t,fragile]\frametitle{Příkazy k ovládanání běhu cyklu} 
  \begin{itemizex}
    \item Pomocí dvou příkazů můžeme ovlivnit běh cyklu.
    \item Příkaz \texttt{break} okmažitě ukončuje cyklus.
    \item Příkaz \texttt{continue} okamžitě přechází na další iteraci.
  \end{itemizex}
\end{frame}



\begin{frame}[t,fragile]\frametitle{Příklad na příkaz \texttt{break}} 
\begin{itemizex}
  \item Vypiš všechny druhé mocniny (přirozených čísel), které jsou menší než $max$ 
\end{itemizex}
\begin{minted}[linenos=true]{c} 
int i, mocnina, max = 100;

for (i = 1; ; i++) {
    mocnina = i * i;
    if (mocnina < max) 
        printf("%i ", mocnina);
    else
        break;
}
// 1 4 9 16 25 36 49 64 81
\end{minted}
\end{frame}


\begin{frame}[t,fragile]\frametitle{Příklad na \texttt{continue}} 
\begin{itemizex}
  \item Vypiš všechna sudá čísla, která jsou menší než $n$ a která nejsou dělitelná šesti
\begin{minted}[linenos=true]{c} 
int i, n = 20;

for (i = 0; i < n; i += 2) {
    if (i % 6 == 0)
        continue;
    printf("%i ", i);
}
// 2 4 8 10 14 16 
\end{minted}
\end{itemizex}
\end{frame}

\begin{frame}[t,fragile]{Úlohy}
\begin{center}
\vskip 1cm
{\Large Seznam úloh je na \url{http://KMIup1.jdem.cz}}
\vskip 2cm
\url{http://tux.inf.upol.cz/~havrlant/}
\end{center}
\end{frame}


\end{document}