\documentclass{beamer}
\usetheme[pageofpages=z,
          bullet=circle,
          titleline=true,
          alternativetitlepage=true,
          titlepagelogo=uplogo,
          ]{UVT}

\usepackage[utf8x]{inputenc}
\usepackage{czech}
\usepackage{minted}
\usepackage{color}
\usepackage{amssymb}
\usepackage{array}
\usepackage{graphicx}
\newcolumntype{C}[1]{>{\centering\let\newline\\\arraybackslash\hspace{0pt}}m{#1}}

  \newenvironment{itemize4}%
  {\large \begin{itemize}%
    \setlength{\itemsep}{4pt}%
    \setlength{\parskip}{4pt}}%
  {\end{itemize}}

\newenvironment{itemizey}%
  {\large \begin{itemize}%
    \setlength{\itemsep}{6pt}%
    \setlength{\parskip}{6pt}}%
  {\end{itemize}}

% \usepackage{url}
\newcommand{\celllength}{2mm}

\newenvironment{itemizex}%
  {\large \begin{itemize}%
    \setlength{\itemsep}{8pt}%
    \setlength{\parskip}{8pt}}%
  {\end{itemize}}

\newenvironment{enumeratex}%
  {\large \begin{enumerate}%
    \setlength{\itemsep}{6pt}%
    \setlength{\parskip}{6pt}}%
  {\end{enumerate}}


\title{Základy programování 1 (KMI/ZP1 a KMI/UP1)}
\subtitle{Cvičení 10: Dynamická alokace paměti}
\author{Lukáš Havrlant}
\date{21. listopadu 2012}
\institute{Univerzita Palackého}

\begin{document}

\begin{frame}[t,plain]
\titlepage
\end{frame}


\begin{frame}[t,fragile]\frametitle{Ukazatel: opakování} 
    \begin{itemizex}
        \item Ukazatel = proměnná, která uchovává adresu.
        % \item Ukazatel musí mít určen svůj typ, tj. musíme vědět, na co ukazujeme.
        \item Jako první musíme do ukazatele uložit adresu.
    \end{itemizex}

    \begin{minted}[linenos=true]{c}
void inkrementuj(int* x) {
    (*x)++;
}
int main() {
    int cislo = 10;
    int* uk = &cislo;
    inkrementuj(uk);
    printf("%i", cislo); // 11
}    
\end{minted}
\end{frame}


\begin{frame}[t,fragile]\frametitle{Odkaz na strukturu} 
    \begin{itemizex}
        \item Můžeme vytvořit i odkaz na strukturu:
        \begin{minted}[linenos=true]{c} 
typedef struct {
    int citatel, jmenovatel;
} Zlomek;
        \end{minted}
        \item Po vytvoření struktury můžeme vytvořit i ukazatel na ni:
        \begin{minted}[linenos=true]{c} 
Zlomek polovina = {1, 2};
Zlomek* ukpol = &polovina;
        \end{minted}
    \end{itemizex}
\end{frame}


\begin{frame}[t,fragile]\frametitle{Člen struktury} 
    \begin{itemizex}
        \item Pokud máme ukazatel na strukturu, ke členům se dostaneme klasicky přes hvězdičku:
        \begin{minted}[linenos=true]{c} 
Zlomek polovina = {1, 2};
Zlomek* ukpol = &polovina;
(*ukpol).citatel = 0;
(*ukpol).jmenovatel = -8;
        \end{minted}
        \item Můžeme využít zkratku a místo \texttt{(*ukpol).citatel} psát \texttt{ukpol->citatel}
        \begin{minted}[linenos=true]{c} 
ukpol->citatel = 0;
ukpol->jmenovatel = -8;
        \end{minted}
    \end{itemizex}
\end{frame}


\begin{frame}[t,fragile]\frametitle{Struktura obsahující sebe samu} 
    \begin{itemize}
        \item Pokud definujeme členy struktury, nemůžeme jako jeden z členů zvolit tutéž strukturu:
        \begin{minted}[linenos=true]{c} 
typedef struct {
    int hodnota;
    struct Seznam dalsi; // Toto nejde
} Seznam;
        \end{minted}
        \item Obejdeme to ukazatelem na strukturu:
        \begin{minted}[linenos=true]{c} 
typedef struct szn {
    int hodnota;
    struct szn* dalsi; // Toto jde
} Seznam;
        \end{minted}
        \item Typ \texttt{szn} používáme jen uvnitř struktury. Všude jinde pak používáme typ \texttt{Seznam}.
    \end{itemize}
\end{frame}



\begin{frame}[t,fragile]\frametitle{Lokální proměnné funkce} 
    \begin{itemize}
        \item Po ukončení funkce zanikají všechny lokální proměnné v dané funkci.
        \item Nemůžeme vrátit ukazatel na lokální proměnnou funkce!
        \begin{minted}[linenos=true]{c} 
int* funkce(int a, int b) {
    int c = a + b;
    int pole[] = {a, b};
    double d = (double)a / b;
    return &c;    // NEJDE!!!
    return &a;    // NEJDE!!!
    return pole;  // NEJDE!!!
}
        \end{minted}
        \item Na pátém řádku funkce končí a všechny proměnné \texttt{a}, \texttt{b}, \texttt{c}, \texttt{pole}, \texttt{d} zanikají.
    \end{itemize}
\end{frame}


% \begin{frame}[t,fragile]\frametitle{Ukazatel na lokální proměnnou funkce} 
%     \begin{itemizex}
%         \item Nemůžeme vrátit ukazatel na lokální proměnnou funkce!
%         \begin{minted}[linenos=true]{c} 

%         \end{minted}
%         \item Všechny proměnné zanikají, takže vracíme ukazatel, který ukazuje na proměnnou, která zanikla.
%     \end{itemizex}
% \end{frame}


\begin{frame}[t,fragile]\frametitle{Operátor \texttt{sizeof}} 
    \begin{itemizex}
        \item Operátor \texttt{sizeof(typ)} vrací velikost typu v bajtech.
        \item Návratová hodnota je typu \texttt{size\_t}, celočíselný neznaménkový typ. Modifikátor v \texttt{printf} je \texttt{\%zu}
        \begin{minted}[linenos=true]{c} 
int cislo = 17;
printf("%zu\n", sizeof(char));   // 1
printf("%zu\n", sizeof(short));  // 2
printf("%zu\n", sizeof(int));    // 4
printf("%zu\n", sizeof(Zlomek)); // 8
printf("%zu\n", sizeof(int*));   // 8
printf("%zu\n", sizeof(char*));  // 8
printf("%zu\n", sizeof(cislo));  // 4
        \end{minted}
    \end{itemizex}
\end{frame}


\begin{frame}[t,fragile]\frametitle{Ukazatel na \texttt{void}} 
    \begin{itemizex}
        \item Beztypový ukazatel: \texttt{void*}
        \item Takový ukazatel může ukazovat na libovolný typ.
        \item Ale nemůžeme ho dereferencovat. 
        \begin{minted}[linenos=true]{c} 
int cislo = 5;
void* ukaz = &cislo;
*ukaz = 7; // NEJDE!!!
*((int*)ukaz) = 7;
        \end{minted}
    \end{itemizex}
\end{frame}


\begin{frame}[t,fragile]\frametitle{Dynamické alokování paměti} 
    \begin{itemizex}
        \item Funkce \texttt{malloc} slouží k dynamickému alokování paměti.
        \item Má jeden parametr -- počet bajtů, které má alokovat.
        \item \texttt{malloc} vrací ukazatel na kus alokované paměti. Pokud alokace selže, vrací \texttt{NULL}. Ukazatel je typu \texttt{void*}, takže bychom ho měli přetypovat:
            \begin{minted}[linenos=true]{c} 
int* pole;
pole = malloc(40); // NEPOUZIVAME! Alokuje 40 bajtu
pole = malloc(10 * sizeof(int)); // Spravne
pole = (int*) malloc(10 * sizeof(int)); // Lepsi
pole[0] = 10;
pole[9] = 42;
            \end{minted}
    \end{itemizex}
\end{frame}


\begin{frame}[t,fragile]\frametitle{Dynamicky alokované pole můžeme vracet z funkce} 
    \begin{itemizex}
        \item Funkce \texttt{map2}, která bere pole, zdvojnásobí každý prvek a vrátí nové pole.
    \end{itemizex}

    \begin{minted}[linenos=true]{c} 
int* map2(int* pole, int delka) {
    int i;
    int* nove_pole = (int*)malloc(sizeof(int) * delka);
    // int nove_pole[delka]; -> toto by nefugovalo!!!
    for (i = 0; i < delka; i++) {
        nove_pole[i] = pole[i] * 2;
    }
    return nove_pole;
}
    \end{minted}
\end{frame}


\begin{frame}[t,fragile]\frametitle{Dynamicky alokovanou strukturu taky\dots} 
\begin{minted}[linenos=true]{c} 
Zlomek* vytvor_zlomek(int cit, int jm) {
    // Zlomek zlomek = {cit, jm}; 
    // return &zlomek; // NEJDE!!!
    Zlomek* zlomek = (Zlomek*)malloc(sizeof(Zlomek));
    zlomek->citatel = cit;
    zlomek->jmenovatel = jm;
    return zlomek;
}
\end{minted}
\end{frame}


\begin{frame}[t,fragile]\frametitle{Alokace paměti pro řetězec} 
    \begin{itemizex}
        \item Pokud například chceme nakopírovat řetězec \texttt{"ahoj"} do jiné proměnné, musíme alokovat místo pro 5 znaků!
        \item \texttt{"ahoj"} = \texttt{\{'a', 'h', 'o', 'j', '\textbackslash0'\}}
        \item Když chceme spojit řetězce \texttt{"xy"} a \texttt{"123"} do jednoho, musíme alokovat místo pro 6 znaků.
    \end{itemizex}
\end{frame}


\begin{frame}[t,fragile]\frametitle{Další funkce} 
    \begin{itemizex}
        \item Funkce \texttt{void *calloc(size\_t items, size\_t size)} alokuje paměť pro items prvků o velikosti size (tedy items*size bytů), navíc tuto paměť vyplní nulami.
        \item Funkce \texttt{void *realloc(void* old, size\_t size)} zajišťující změnu velikosti dříve alokovaného bloku paměti old na velikost size. Data původního bloku paměti budou zkopírována do nově alokovaného.
        \item Všechny zmiňované funkce jsou v knihovně \texttt{stdlib.h}
    \end{itemizex}
\end{frame}


\begin{frame}[t,fragile]\frametitle{Uvolnění paměti} 
    \begin{itemizex}
        \item Pokud už nepotřebujeme část paměti, měli bychom ji uvolnit.
        \item K tomu slouží funkce \texttt{free}.
        \begin{minted}[linenos=true]{c} 
int* cisla = (int*) calloc(5, sizeof(int));
// udelej neco s polem "cisla"
//OK, pole cisla uz nepotrebujeme
free(cisla);
        \end{minted}
    \end{itemizex}
\end{frame}


\begin{frame}[t,fragile]\frametitle{Memory leak} 
    \begin{itemizex}
        \item Pokud alokujeme paměť pomocí \texttt{malloc}/\texttt{calloc}, měli bychom tuto paměť uvolnit pomocí \texttt{free}.
        \item \dots jinak se paměť uvolní až při skončení celého programu.
        \begin{minted}[linenos=true]{c} 
void vytvor_memory_leak(int a) {
    char* pole = (char*) malloc(a * sizeof(char));
}

int main() {
    // Vytvor memory leak o velikost 1 MiB
    vytvor_memory_leak(1024 * 1024);
    jina_funkce();
}
        \end{minted}
    \end{itemizex}
\end{frame}


\begin{frame}[t,fragile]\frametitle{Zásobník} 
    \begin{itemizex}
        \item Má tři základní funkce: \texttt{pridej} pro přidání prvku do zásobníku, \texttt{vrchol}, který vrací prvek na vrcholu zásobníku a \texttt{odeber}, který odebere prvek z vrcholu.
        \begin{figure}[htb]
            \centering
            \includegraphics[scale=0.7]{img/zasobnik.pdf}
        \end{figure}
        \item {\tiny Zdroj: A. Večerka, Základní algoritmy}
    \end{itemizex}
\end{frame}


\begin{frame}[t,fragile]\frametitle{K čemu je to dobré} 
    \begin{itemizey}
        \item Jak zjistit, zda ve výrazu \uv{sedí} závorky?
        \item Pokud narazíme na otevírací závorku, přidáme ji na zásobník. Pokud narazíme na uzavírací závorku, odebereme z vrcholu zásobníku otevírací závorku a ta musí odpovídat.
        \item Příklad pro řetězec \texttt{(a+[b*(c+2)]*3)}:
\begin{verbatim}
prazdny     // pridame (
(           // pridame [
([          // pridame (
([(         // ( sedi s ) -> odebereme (
([          // [ sedi s ] -> odebereme [
(           // ( sedi s ) -> odebereme (
prazdny     // zas. je prazdy, zavorky jsou OK
\end{verbatim}
    \end{itemizey}
\end{frame}


\begin{frame}[t,fragile]\frametitle{Závorky znova} 
    \begin{itemizex}
        \item Co třeba řetězec \texttt{(x+[5*(x-4]))}?
        \begin{verbatim}
prazdny     // pridame (
(           // pridame [
([          // pridame (
([(         // ] nesedi s ( -> zavorky nesedi
        \end{verbatim}
    \end{itemizex}
\end{frame}


\begin{frame}[t,fragile]\frametitle{Jak to naprogramovat?} 
    \begin{itemizex}
        \item Pomocí pole: dělali jste v algmatu.
        \item Pomocí spojového seznamu: máte za úkol.
        \item Princip: definujeme strukturu s členy: \texttt{hodnota} (tam bychom např. ukládali závorky) a \texttt{predchozi} (ukazatel na předchozí prvek / na předchozí strukturu).
    \end{itemizex}
\end{frame}

\begin{frame}[t,fragile]\frametitle{Přidání prvku: přidáme 5 a pak 7} 
\begin{figure}[htb]
    \centering
    \includegraphics[scale=0.9]{img/zas-1.pdf}
\end{figure}
\end{frame}


\begin{frame}[t,fragile]\frametitle{Odebrání prvku} 
\begin{figure}[htb]
    \centering
    \includegraphics[scale=0.9]{img/zas-2.pdf}
\end{figure}
\end{frame}


\begin{frame}[t,fragile]\frametitle{Definice struktury a přidání prvku} 
\begin{minted}[linenos=true]{c} 
typedef struct Zasob {
    int hodnota;
    struct Zasob* predchozi;
} prvek;

int main() {
    prvek* vrchol = NULL;
    prvek* pet = (prvek*) malloc(sizeof(prvek));
    pet->hodnota = 5;
    pet->predchozi = NULL;
    vrchol = pet;
    printf("hodnota na vrcholu:%i\n", vrchol->hodnota); //5
}
\end{minted}
\end{frame}


\begin{frame}[t,fragile]\frametitle{Obecný postup} 
    \begin{itemizex}
        \item Funkce \texttt{pridej} musí vytvořit novou strukturu, uložit do ní předanou hodnotu, uložit do členu \texttt{predchozi} odkaz na strukturu, která je aktuálně na vrcholu zásobníku a vrátit odkaz na nově vytvořenou strukturu.
        \item Funkce \texttt{odeber} musí zrušit strukturu, která je vrcholu a vrátit odkaz na předchozí strukturu.
        \item Funkce \texttt{vrchol} jen vrátí hodnotu ze struktury, která je na vrcholu.
    \end{itemizex}
\end{frame}


\begin{frame}[t,fragile]{Úlohy}
\begin{center}
\vskip 1cm
{\Large Seznam úloh je na \url{http://KMIup1.jdem.cz}}
\vskip 2cm
\url{http://tux.inf.upol.cz/~havrlant/}
\end{center}
\end{frame}


\end{document}