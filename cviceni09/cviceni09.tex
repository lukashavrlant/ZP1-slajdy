\documentclass{beamer}
\usetheme[pageofpages=z,
          bullet=circle,
          titleline=true,
          alternativetitlepage=true,
          titlepagelogo=uplogo,
          ]{UVT}

\usepackage[utf8x]{inputenc}
\usepackage{czech}
\usepackage{minted}
\usepackage{color}
\usepackage{amssymb}
\usepackage{array}
\usepackage{graphicx}
\newcolumntype{C}[1]{>{\centering\let\newline\\\arraybackslash\hspace{0pt}}m{#1}}

% \usepackage{url}
\newcommand{\celllength}{2mm}

\newenvironment{itemizex}%
  {\large \begin{itemize}%
    \setlength{\itemsep}{8pt}%
    \setlength{\parskip}{8pt}}%
  {\end{itemize}}

\newenvironment{enumeratex}%
  {\large \begin{enumerate}%
    \setlength{\itemsep}{6pt}%
    \setlength{\parskip}{6pt}}%
  {\end{enumerate}}


\title{Základy programování 1 (KMI/ZP1 a KMI/UP1)}
\subtitle{Cvičení 9: Ukazatele}
\author{Lukáš Havrlant}
\date{14. listopadu 2012}
\institute{Univerzita Palackého}

\begin{document}

\begin{frame}[t,plain]
\titlepage
\end{frame}


\begin{frame}[t,fragile]\frametitle{Paměť počítače} 
    \begin{figure}[htb]
    \centering
    \includegraphics{img/pamet1-2.pdf}
    \end{figure}

    \begin{minted}[linenos=true]{c} 
char znak = '!';
int cislo = 1073741824;
    \end{minted}

    \begin{figure}[htb]
    \centering
    \includegraphics{img/pamet2-2.pdf}
    \end{figure}
\end{frame}


\begin{frame}[t,fragile]\frametitle{Adresa proměnné -- operátor reference} 
\begin{minted}[linenos=true]{c} 
char znak = '!';
int cislo = 1073741824; 
adresa = &cislo; // 43
adresa = &znak;  // 41
\end{minted}

\begin{figure}[htb]
    \centering
    \includegraphics[scale=0.85]{img/pamet2-2.pdf}
\end{figure}

\begin{itemizex}
    \item Ampersand \texttt{\&} je operátor \textit{reference}. Vrací \textit{adresu} proměnné.
    \item \texttt{\&znak} vrací adresu, na které je uložena hodnota proměnné \texttt{znak}, tj. číslo $41$.
\end{itemizex}
\end{frame}


\begin{frame}[t,fragile]\frametitle{Čtení hodnoty na dané adrese} 
\begin{figure}[htb]
    \centering
    \includegraphics[scale=0.75]{img/pamet3-2.pdf}
\end{figure}

\begin{itemize}
    \item V proměnné \texttt{adresa} máme uloženou adresu jiné proměnné -- proměnné \texttt{znak}.
    \item Jak přečíst hodnotu, která se nachází na adrese, kterou máme uloženou v proměnné \texttt{adresa}?
    \item Pomocí operátoru \textit{dereference}: \texttt{*}
    \item \texttt{*adresa} vrátí hodnotu buňky s adresou $41$, tj. \uv{!}
    \item {\tiny Poznámka: adresa se běžně nevleze do jedné buňky, tj. do 1 B. Ale 4/8B adresa by se už nevešla na slajdy.}
\end{itemize}
% \begin{figure}[htb]
%     \centering
%     \includegraphics[scale=0.85]{img/pamet3-1.pdf}
% \end{figure}
\end{frame}



\begin{frame}[t,fragile]\frametitle{Čtení typu \texttt{int}} 
\begin{minted}[linenos=true]{c} 
int cislo = 1073741824; 
adresa = &cislo;
\end{minted}

\begin{figure}[htb]
    \centering
    \includegraphics[scale=0.75]{img/pamet3-1.pdf}
\end{figure}

\begin{itemizex}
    \item Na co by se mělo vyhodnotit \texttt{*adresa}?
    \item Musíme zavést nový typ \uv{ukazatel na [typ]}. (Anglicky: pointer)
\end{itemizex}
\end{frame}


\begin{frame}[t,fragile]\frametitle{Ukazatel na typ} 
\begin{minted}[linenos=true]{c} 
char znak = '!';
int cislo = 1073741824;

char *adresaz = &znak; // ukazatel na char
int *adresac = &cislo; // ukazatel na int
\end{minted}
\begin{figure}[htb]
    \centering
    \includegraphics[scale=0.75]{img/pamet3-3.pdf}
\end{figure}
\end{frame}


\begin{frame}[t,fragile]\frametitle{Zápis hodnoty} 
\begin{minted}[linenos=true]{c} 
char znak = '!';
int cislo = 1073741824;
char *adresaz = &znak;
int *adresac = &cislo;
*adresac = 666;
*adresaz = '?';
printf("%c, %i\n", znak, cislo);
\end{minted}

\begin{figure}[htb]
    \centering
    \includegraphics[scale=0.75]{img/pamet3-4.pdf}
\end{figure}
\end{frame}


\begin{frame}[t,fragile]{Úlohy}
\begin{center}
\vskip 1cm
{\Large Seznam úloh je na \url{http://KMIup1.jdem.cz}}
\vskip 2cm
\url{http://tux.inf.upol.cz/~havrlant/}
\end{center}
\end{frame}


\end{document}