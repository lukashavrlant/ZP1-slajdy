\documentclass{beamer}
\usetheme[pageofpages=z,
          bullet=circle,
          titleline=true,
          alternativetitlepage=true,
          titlepagelogo=uplogo,
          ]{UVT}

\usepackage[utf8x]{inputenc}
\usepackage{czech}
\usepackage{minted}
\usepackage{color}
\usepackage{amssymb}
\usepackage{array}
\usepackage{graphicx}
\newcolumntype{C}[1]{>{\centering\let\newline\\\arraybackslash\hspace{0pt}}m{#1}}

  \newenvironment{itemize4}%
  {\large \begin{itemize}%
    \setlength{\itemsep}{4pt}%
    \setlength{\parskip}{4pt}}%
  {\end{itemize}}

\newenvironment{itemizey}%
  {\large \begin{itemize}%
    \setlength{\itemsep}{6pt}%
    \setlength{\parskip}{6pt}}%
  {\end{itemize}}

% \usepackage{url}
\newcommand{\celllength}{2mm}

\newenvironment{itemizex}%
  {\large \begin{itemize}%
    \setlength{\itemsep}{8pt}%
    \setlength{\parskip}{8pt}}%
  {\end{itemize}}

\newenvironment{enumeratex}%
  {\large \begin{enumerate}%
    \setlength{\itemsep}{6pt}%
    \setlength{\parskip}{6pt}}%
  {\end{enumerate}}


\title{Základy programování 1 (KMI/ZP1 a KMI/UP1)}
\subtitle{Cvičení 9: Ukazatele}
\author{Lukáš Havrlant}
\date{14. listopadu 2012}
\institute{Univerzita Palackého}

\begin{document}

\begin{frame}[t,plain]
\titlepage
\end{frame}


\begin{frame}[t,fragile]\frametitle{Paměť počítače} 
    \begin{figure}[htb]
    \centering
    \includegraphics{img/pamet1-2.pdf}
    \end{figure}

    \begin{minted}[linenos=true]{c} 
char znak = '!';
int cislo = 1073741824;
    \end{minted}

    \begin{figure}[htb]
    \centering
    \includegraphics{img/pamet2-2.pdf}
    \end{figure}
\end{frame}


\begin{frame}[t,fragile]\frametitle{Adresa proměnné -- operátor reference} 
\begin{minted}[linenos=true]{c} 
char znak = '!';
int cislo = 1073741824; 
adresa = &cislo; // 43
adresa = &znak;  // 41
\end{minted}

\begin{figure}[htb]
    \centering
    \includegraphics[scale=0.85]{img/pamet2-2.pdf}
\end{figure}

\begin{itemizex}
    \item Ampersand \texttt{\&} je operátor \textit{reference}. Vrací \textit{adresu} proměnné.
    \item \texttt{\&znak} vrací adresu, na které je uložena hodnota proměnné \texttt{znak}, tj. číslo $41$.
\end{itemizex}
\end{frame}


\begin{frame}[t,fragile]\frametitle{Čtení hodnoty na dané adrese} 
\begin{figure}[htb]
    \centering
    \includegraphics[scale=0.75]{img/pamet3-2.pdf}
\end{figure}

\begin{itemize}
    \item V proměnné \texttt{adresa} máme uloženou adresu jiné proměnné -- proměnné \texttt{znak}.
    \item Jak přečíst hodnotu, která se nachází na adrese, kterou máme uloženou v proměnné \texttt{adresa}?
    \item Pomocí operátoru \textit{dereference}: \texttt{*}
    \item \texttt{*adresa} vrátí hodnotu buňky s adresou $41$, tj. \uv{!}
    \item {\tiny Poznámka: adresa se běžně nevleze do jedné buňky, tj. do 1 B. Ale 4/8B adresa by se už nevešla na slajdy.}
\end{itemize}
% \begin{figure}[htb]
%     \centering
%     \includegraphics[scale=0.85]{img/pamet3-1.pdf}
% \end{figure}
\end{frame}



\begin{frame}[t,fragile]\frametitle{Čtení typu \texttt{int}} 
\begin{minted}[linenos=true]{c} 
int cislo = 1073741824; 
adresa = &cislo;
\end{minted}

\begin{figure}[htb]
    \centering
    \includegraphics[scale=0.75]{img/pamet3-1.pdf}
\end{figure}

\begin{itemizex}
    \item Na co by se mělo vyhodnotit \texttt{*adresa}?
    \item Musíme zavést nový typ \uv{ukazatel na [typ]}. (Anglicky: pointer)
\end{itemizex}
\end{frame}


\begin{frame}[t,fragile]\frametitle{Ukazatel na typ} 
\begin{itemize4}
    \item Ukazatel (angl. pointer) je datový typ, který slouží k uložení adresy v paměti počítače. 
    \item Jako první musíme do ukazatele uložit onu adresu. 
\end{itemize4}
\begin{minted}[linenos=true]{c} 
char znak = '!';
int cislo = 1073741824;

char* adresaz = &znak; // ukazatel na char
int* adresac = &cislo; // ukazatel na int
\end{minted}
\begin{figure}[htb]
    \centering
    \includegraphics[scale=0.75]{img/pamet3-3.pdf}
\end{figure}
\end{frame}


\begin{frame}[t,fragile]\frametitle{Zápis hodnoty} 
\begin{minted}[linenos=true]{c} 
char* adresaz; 
adresaz = &znak;
*adresaz = '?';

int* adresac; 
adresac = &cislo;
*adresac = 666;
printf("%c, %i\n", znak, cislo);
\end{minted}

\begin{figure}[htb]
    \centering
    \includegraphics[scale=0.75]{img/pamet3-4.pdf}
\end{figure}
\end{frame}



\begin{frame}[t,fragile]\frametitle{Zápis hodnoty podruhé} 
\begin{itemizex}
    \item Pokud neinicializujeme adrese ukazatele, může dojít k segmentation fault.
\end{itemizex}
\begin{minted}[linenos=true]{c} 
char* adresa;
*adresa = 10;
\end{minted}

\begin{figure}[htb]
    \centering
    \includegraphics[scale=0.75]{img/pamet8-1.pdf}
\end{figure}
\end{frame}


\begin{frame}[t,fragile]\frametitle{Shrnutí syntaxe} 
    \begin{itemizex}
        \item \texttt{int cislo;} deklaruje obyčejnou proměnnou typu \texttt{int}.
        \item \texttt{\&cislo} vrací adresu proměnné \texttt{cislo}.
        \item \texttt{int* uk;} deklaruje ukazatel na typ \texttt{int}. 
        \item Do ukazatele musíme vložit adresu, na kterou má ukazovat. To děláme bez hvězdičky: \texttt{uk = \&cislo;}
        \item Hodnotu na dané adrese čteme pomocí *: \texttt{int fn = *uk;}
        \item Hodnotu na danou adresu zapíšeme pomocí *: \texttt{*uk = 10;}
    \end{itemizex}
\end{frame}


\begin{frame}[t,fragile]\frametitle{Použití ukazatele} 
\begin{itemizex}
    \item Pokud máme ukazatel \texttt{*uk} na proměnnou \texttt{cislo}, tak \texttt{*uk} můžeme používat v jakémkoliv kontextu, ve kterém se může objevit \texttt{cislo}.
\begin{minted}[linenos=true]{c} 
int cislo = 10;
int* uk; 
uk = &cislo;
printf("%i\n", cislo + 5 + *uk); // 25
printf("%i\n", *uk * 5); // 50
(*uk)++;
printf("%i, %i\n", *uk, cislo); // 11, 11
\end{minted}
\end{itemizex}
\end{frame}


\begin{frame}[t,fragile]\frametitle{Předávání hodnotou a odkazem} 
    \begin{itemizey}
        \item Pokud předáváme funkci nějaké argumenty, předávají se hodnotou.
        \item Tj. hodnoty z argumentů se nakopírují do nových proměnných, které odpovídají parametrům funkce.
        \item Funkce tam nemůže změnit původní proměnné. 
        \item Řešení: nepředáme hodnotu, ale odkaz na proměnné. 
        \item Pamatujete na \texttt{scanf}? 
        \begin{minted}[linenos=true]{c} 
scanf("%i", &cislo);
        \end{minted}
    \end{itemizey}
\end{frame}


\begin{frame}[t,fragile]\frametitle{Nefunkční funkce na prohození proměnných} 
    \begin{minted}[linenos=true]{c} 
void vymen(char a, char b) {
    char temp = a;
    a = b;
    b = temp;
}

int main() {
    char delka = 10, vyska = 20;
    vymen(delka, vyska);
}
    \end{minted}

\begin{figure}[htb]
    \centering
    \includegraphics[scale=0.75]{img/pamet4-1.pdf}
\end{figure}
\end{frame}


\begin{frame}[t,fragile]\frametitle{Funkční funkce na prohození proměnných} 
\begin{minted}[linenos=true]{c} 
void vymen(char* a, char* b) {
    char temp = *a;
    *a = *b;
    *b = temp;
}

int main() {
    char delka = 10, vyska = 20;
    vymen(&delka, &vyska);
}
\end{minted}

\begin{figure}[htb]
    \centering
    \includegraphics[scale=0.75]{img/pamet4-2.pdf}
\end{figure}
\end{frame}


% \begin{frame}[t,fragile]\frametitle{Jak pracovat s ukazatelem} 
% \begin{minted}[linenos=true]{c} 
% char vek = 25;    
% char *uk_vek, *uk_auto; // prvni obrazek
% uk_vek = &vek; // *uk_vek = 40; -> vek = 40
% *uk_auto = 69; // druhy obrazek
% \end{minted}

% \begin{figure}[htb]
%     \centering
%     \includegraphics[scale=0.75]{img/pamet5-1.pdf}
% \end{figure}

% \begin{figure}[htb]
%     \centering
%     \includegraphics[scale=0.75]{img/pamet5-2.pdf}
% \end{figure}
% \end{frame}



\begin{frame}[t,fragile]\frametitle{Aritmetika ukazatelů} 
    \begin{itemizex}
        \item Adresa je obyčejné číslo, takže s ním můžeme (skoro) normálně počítat.
        \item K adrese můžeme přičítat/odečítat jiné hodnoty.
        \item Důležitý je typ ukazatele. Jinak se počítá se ukazatelem na \texttt{char} než na \texttt{short} atp. 
        \item Pokud napíšeme \texttt{ukazatel+1}, chceme tím získat adresu, kde se nachází další proměnná se stejným typem.
    \end{itemizex}
\end{frame}



\begin{frame}[t,fragile]\frametitle{Aritmetika ukazatelů: příklad} 
\begin{minted}[linenos=true]{c} 
char p[] = {2, 4, 8};
short c[] = {500, 600};
char* ukp = &p[0];
short* ukc = &c[0];
printf("%i\n", *(ukp+1)); // 4
printf("%i\n", *(ukc+1)); // 600
// Jakoby udelal: ukc+(1*velikost_short)
\end{minted}

\begin{figure}[htb]
    \centering
    \includegraphics[scale=0.75]{img/pamet7-1.pdf}
\end{figure}
\end{frame}

\begin{frame}[t,fragile]\frametitle{Ukazatele a pole} 
    \begin{itemizex}
        \item Více méně platí pole = ukazatel.
        \item Identifikátor pole se chová jako konstantní ukazatel na první prvek pole.
    \end{itemizex}

\begin{minted}[linenos=true]{c} 
char p[] = {2, 4, 8, 16, 32};
char* st = p;
printf("%i, %i, %i\n", p[2], *(p+2), *(st+2)); // 8, 8, 8
// p[i] je ekvivalentni *(p+i)
\end{minted}
\begin{figure}[htb]
    \centering
    \includegraphics[scale=0.75]{img/pamet6-1.pdf}
\end{figure}
\end{frame}

\begin{frame}[t,fragile]\frametitle{Pole jako ukazatel} 
    \begin{itemizex}
        \item Pokud chceme předat funkci jako argument pole, předáme ve skutečnosti ukazatel na první prvek.
        \item Tato volání funkce \texttt{obrat\_pole} by byla identická:
\begin{minted}[linenos=true]{c} 
int pole = {2, 5, 12, 18};
obrat_pole(pole, 4);
obrat_pole(&pole[0], 4);
\end{minted}
    \item Uvnitř funkce \texttt{obrat\_pole} už tak vidíme jen ukazatel na první prvek pole.
    \end{itemizex}
\end{frame}


\begin{frame}[t,fragile]\frametitle{Ukazatel jako pole} 
    \begin{itemizex}
        \item Uvnitř funkce vidíme místo pole ukazatel na první prvek.
        \item S tímto ukazatelem ale můžeme pracovat jako s polem.
        \begin{minted}[linenos=true]{c} 
void funkce(int* pole) {
    printf("%i, %i\n", pole[1], *(pole + 1));
}

int main() {
    int pole[] = {3, 6, 9};
    funkce(pole);       // 6, 6
    funkce(&pole[0]);   // 6, 6
}
        \end{minted}
    \end{itemizex}
\end{frame}


\begin{frame}[t,fragile]\frametitle{Jak předat část pole} 
\begin{itemizex}
    \item Pokud předáme funkci ukazatel na třetí prvek pole, tak volaná funkce bude brát tento třetí prvek jako svůj první.
\end{itemizex}

\begin{minted}[linenos=true]{c} 
int pole[] = {0, 3, 6, 9, 12, 15}, delka = 6;
obrat_pole(pole, delka);
vypis_pole(pole, delka); // 15 12 9 6 3 0
obrat_pole(&pole[2], 4);
vypis_pole(pole, delka); // 15 12 0 3 6 9 
\end{minted}

\begin{itemizex}
    \item Předali jsme funkci ukazatel na třetí prvek.
    \item Funkce tak \uv{viděla} jen část pole: \texttt{\{9, 6, 3, 0\}}.
\end{itemizex}
\end{frame}


\begin{frame}[t,fragile]\frametitle{Jak vracet část pole} 
    \begin{itemizex}
        \item Pole můžeme vrátit tak, že vrátíme ukazatel na nultý prvek.
        \item Pokud vrátíme ukazatel na $i$-tý prvek, pole \uv{ořežeme}.
    \end{itemizex}

    \begin{minted}[linenos=true]{c} 
int* orez_pole(int* pole, int kolik) {
  return &pole[kolik];
}

int main() {
  int pole[] = {0, 3, 6, 9, 12, 15}, delka = 6;
  vypis_pole(orez_pole(pole, 2), delka - 2); // 6 9 12 15 
  vypis_pole(pole, delka);                 // 0 3 6 9 12 15 
  vypis_pole(orez_pole(pole, 5), delka - 5); // 15 
}
    \end{minted}
\end{frame}


\begin{frame}[t,fragile]\frametitle{Nulová adresa} 
    \begin{itemizex}
        \item Nulová adresa je vždy neplatná.
        \item Používáme konstantu \texttt{NULL}.
        \item Můžeme ji použít pro signalizaci toho, že ukazatel \uv{nikam neodkazuje}.
    \end{itemizex}

\begin{minted}[linenos=true]{c} 
int* ukazatel;
ukazatel = 0; 
ukazazel = NULL; // lepsi
if (ukazatel) {
    // splneno pri nenulove adrese
}
\end{minted}
\end{frame}

\begin{frame}[t,fragile]\frametitle{Rozděl pole} 
\begin{minted}[linenos=true]{c} 
int* rozdel_pole(int* pole, int delka, int oddelovac) {
    int i;
    for (i = 0; i < delka; i++)
        if (pole[i] == oddelovac)
            return &pole[i];
    return NULL;
}

int main() {
    int pole[] = {0, 3, 6, 9, 12, 15}, delka = 6;
    int* pole1 = rozdel_pole(pole, delka, 9); // {9, 12, 15}
    int* pole2 = rozdel_pole(pole, delka, 12); // {12, 15}
    int* pole3 = rozdel_pole(pole, delka, 5); // NULL
}
\end{minted}
\end{frame}


\begin{frame}[t,fragile]\frametitle{Shrnutí} 
    \begin{itemize4}
        \item Operátor refenrence \texttt{\&} vrací adresu proměnné.
        \item Operátor dereference \texttt{*} zjistí hodnotu proměnné na dané adrese. 
        \item Pokud deklarujeme nový ukazatel pomocí \texttt{int *ukaz;}, musíme do něj později uložit adresu, na kterou má ukazovat.
        \item Pokud má funkce vracet ukazatel, přidáme za typ hvězdičku: \texttt{int* vrat\_ukazatel() \{...\}}
        \item Identifikátor pole slouží zároveň jako ukazatel na první prvek pole. 
        \item Ukazatel můžeme indexovat stejně jako pole: \texttt{*(pole+3)} a \texttt{pole[3]} je stejné. \texttt{*(ukaz+3)} a \texttt{ukaz[3]} také. 
    \end{itemize4}
\end{frame}


\begin{frame}[t,fragile]{Úlohy}
\begin{center}
\vskip 1cm
{\Large Seznam úloh je na \url{http://KMIup1.jdem.cz}}
\vskip 2cm
\url{http://tux.inf.upol.cz/~havrlant/}
\end{center}
\end{frame}


\end{document}