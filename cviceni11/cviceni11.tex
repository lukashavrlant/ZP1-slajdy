\documentclass{beamer}
\usetheme[pageofpages=z,
          bullet=circle,
          titleline=true,
          alternativetitlepage=true,
          titlepagelogo=uplogo,
          ]{UVT}

\usepackage[utf8x]{inputenc}
\usepackage{czech}
\usepackage{minted}
\usepackage{color}
\usepackage{amssymb}
\usepackage{array}
\usepackage{graphicx}
\newcolumntype{C}[1]{>{\centering\let\newline\\\arraybackslash\hspace{0pt}}m{#1}}

  \newenvironment{itemize4}%
  {\large \begin{itemize}%
    \setlength{\itemsep}{4pt}%
    \setlength{\parskip}{4pt}}%
  {\end{itemize}}

\newenvironment{itemizey}%
  {\large \begin{itemize}%
    \setlength{\itemsep}{6pt}%
    \setlength{\parskip}{6pt}}%
  {\end{itemize}}

% \usepackage{url}
\newcommand{\celllength}{2mm}

\newenvironment{itemizex}%
  {\large \begin{itemize}%
    \setlength{\itemsep}{8pt}%
    \setlength{\parskip}{8pt}}%
  {\end{itemize}}

\newenvironment{enumeratex}%
  {\large \begin{enumerate}%
    \setlength{\itemsep}{6pt}%
    \setlength{\parskip}{6pt}}%
  {\end{enumerate}}


\title{Základy programování 1 (KMI/ZP1 a KMI/UP1)}
\subtitle{Cvičení 11: Předání parametru odkazem}
\author{Lukáš Havrlant}
\date{28. listopadu 2012}
\institute{Univerzita Palackého}

\begin{document}

\begin{frame}[t,plain]
\titlepage
\end{frame}


\begin{frame}[t,fragile]\frametitle{Návratová hodnota funkce} 
    \begin{itemizex}
        \item Běžně píšeme funkce tak, aby svůj výsledek vracela jako návratovou hodnotu:
        \begin{minted}[linenos=true]{c} 
int soucet(int a, int b) {
    return a + b;
}

int* prirozena(int n) {
    int* pole = (int*) malloc(n * sizeof(int));
    int i;
    for (i = 0; i < n; i++) 
        pole[i] = i+1;
    return pole;
}
        \end{minted}
    \end{itemizex}
\end{frame}


\begin{frame}[t,fragile]\frametitle{Uložení návratové hodnoty do parametru} 
    \begin{itemizex}
        \item Můžeme ale postupovat i tak, že hodnotu uložíme do parametru, ve kterém máme ukazatel na proměnnou
        \begin{minted}[linenos=true]{c} 
void soucet(int a, int b, int* vysledek) {
    *vysledek = a + b;
}

int main() {
    int x;
    soucet(1, 4, &x);
    printf("%i\n", x);
}
        \end{minted}
    \end{itemizex}
\end{frame}


\begin{frame}[t,fragile]\frametitle{Ukazatel na ukazatel} 
    \begin{itemizex}
        \item Pokud chceme ukládat do proměnné typu int, předáme funkci ukazatel na int
        \item Pokud chceme ukládat do proměnné typu řetězec, předáme funkci ukazatel na řetězec
        \item $\longrightarrow$ řetězec sám je ale ukazatel na char.
        \item $\longrightarrow$ musíme tak předat ukazatel na ukazatel na char!
    \end{itemizex}
\end{frame}

\begin{frame}[t,fragile]\frametitle{Předání řetězce} 
    \begin{minted}[linenos=true]{c} 
void mystrcpy(char* retezec, char** kopie) {
    int delkap = strlen(retezec) + 1;
    *kopie = malloc(delkap);
    int i = 0;
    for (i = 0; i < delkap; i++)
        (*kopie)[i] = retezec[i];
}

int main() {
    char* text = "ahoj";
    char* kopie;
    mystrcpy(text, &kopie);
    printf("%s\n", kopie);
}
    \end{minted}
\end{frame}


\begin{frame}[t,fragile]\frametitle{Modifikátor const} 
    \begin{itemizex}
        \item V C existuje modifikátor \texttt{const}
        \item Dělá z proměnné konstantní proměnnou, tj. proměnnou, kterou můžeme jen číst, nemůžeme do ní zapisovat
        \begin{minted}[linenos=true]{c} 
const int pocet = 10;
pocet = 5; // NEJDE!
        \end{minted}
        \item Můžeme ho použít i před parametrem funkce:
        \begin{minted}[linenos=true]{c} 
char* funkce(const int a, const char* retezec) {
    a = 10; // NEJDE, chyba
    retezec[0] = '!'; // NEJDE, chyba
    return &retezec[0]; // Nemeli bychom, warning
}
        \end{minted}
    \end{itemizex}
\end{frame}


\begin{frame}[t,fragile]\frametitle{Funkce z knihovny \texttt{string.h}} 
    \begin{itemizex}
        \item V knihovně \texttt{string.h} jsou některé užitečné funkce pro práci s řetězci.
        \item \texttt{\#include <string.h>}
        \item Asi nejzásadnější: \texttt{strlen("text")} vrací délku řetězce. 
        \item Přehled funkcí viz třeba \url{http://www.sallyx.org/sally/c/c19.php}
    \end{itemizex}
\end{frame}


\begin{frame}[t,fragile]\frametitle{Nikdy nepoužívejte globální proměnné zbytečně!} 
\begin{minted}[linenos=true]{c} 
int i; // Za tohle by vam meli useknout ruce
// promenna "i" by mela byt definovana v kazde funkci zvlast
void vypis_faktorialy(int pocet) {
    for (i = 1; i <= pocet; i++) {
        printf("%i ", faktorial(i));
    }
}

int faktorial(int n) {
    int fakt = 1;
    for (i = n; i > 0; i--) {
        fakt = fakt * i;
    }
    return fakt;
}
\end{minted}
\end{frame}


\begin{frame}[t,fragile]{Úlohy}
\begin{center}
\vskip 1cm
{\Large Seznam úloh je na \url{http://KMIup1.jdem.cz}}
\vskip 2cm
\url{http://tux.inf.upol.cz/~havrlant/}
\end{center}
\end{frame}


\end{document}