\documentclass{beamer}
\usetheme[pageofpages=z,
          bullet=circle,
          titleline=true,
          alternativetitlepage=true,
          titlepagelogo=uplogo,
          ]{UVT}

\usepackage[utf8x]{inputenc}
\usepackage{czech}
\usepackage{minted}
% \usepackage{url}


\newenvironment{itemizex}%
  {\large \begin{itemize}%
    \setlength{\itemsep}{8pt}%
    \setlength{\parskip}{8pt}}%
  {\end{itemize}}

\newenvironment{enumeratex}%
  {\large \begin{enumerate}%
    \setlength{\itemsep}{6pt}%
    \setlength{\parskip}{6pt}}%
  {\end{enumerate}}


\title{Úvod do programování 1 (KMI/ZP1 a KMI/UP1)}
\subtitle{Cvičení 3: Větvení}
\author{Lukáš Havrlant}
\date{3. října 2012}
\institute{Univerzita Palackého}

\begin{document}

\begin{frame}[t,plain]
\titlepage
\end{frame}



\begin{frame}[t,fragile]\frametitle{Konstrukce \texttt{if}} 
  \begin{itemizex}
    \item Pokud je splněna podmínka $X$, proveď $Y$, jinak proveď $Z$.
    \item Obecná syntaxe: 
    \begin{minted}[linenos=true]{c} 
if (podminka)
  vyraz_splneno;
else 
  vyraz_nespneno;
    \end{minted}
    \item Else větev je nepovinná.
    \item Za \texttt{if}em a \texttt{else}m není středník.
  \end{itemizex}
\end{frame}


\begin{frame}[t,fragile]\frametitle{Jednoduchý příklad} 
  \begin{minted}[linenos=true]{c} 
int cislo = 5;
if (cislo > 0)
  printf("Cislo je kladne");


int jine_cislo = -4;
if (jine_cislo > 0)
  printf("Cislo je kladne");
else
  printf("Cislo je zaporne");
  \end{minted}
\end{frame}



\begin{frame}[t,fragile]\frametitle{Složitější jednoduchý příklad} 
  \begin{minted}[linenos=true]{c} 
int cislo = -5;

if (cislo >= 0)
if (cislo == 0)
printf("Cislo je rovno nule");
else
printf("Cislo je zaporne");
  \end{minted}
\end{frame}


\begin{frame}[t,fragile]\frametitle{Složitější jednoduchý příklad přehledněji} 
\begin{minted}[linenos=true]{c} 
if (cislo >= 0)
    if (cislo == 0)
        printf("Cislo je rovno nule");
    else
        printf("Cislo je kladne");

if (cislo >= 0)
    if (cislo == 0)
        printf("Cislo je rovno nule");
else
    printf("Cislo je kladne");
\end{minted}
\end{frame}


\begin{frame}[t,fragile]\frametitle{Bloky} 
  \begin{itemizex}
    \item Blok umožňuje zapsat více příkazů na místě jednoho
    \begin{minted}[linenos=true]{c} 
{
    int a = 1, b = 2;
    int c = a + b;
    printf("Ahoj svete!");

    if (c = 4)
        printf("Pravda.");
}
    \end{minted}
  \end{itemizex}
\end{frame}


\begin{frame}[t,fragile]\frametitle{Jednoduchý příklad s bloky} 
  \begin{minted}[linenos=true]{c} 
if (cislo >= 0) {
    if (cislo == 0)
        printf("Cislo je rovno nule");
    else 
        printf("Cislo je kladne");
} else {
  printf("Cislo je zaporne");
}
  \end{minted}

  \begin{itemizex}
    \item Jaké je výchozí chování?
    \item Lepší nevědět a používat závorky.
  \end{itemizex}
\end{frame}


\begin{frame}[t,fragile]\frametitle{Složitější příklad s bloky} 
  \begin{minted}[linenos=true]{c} 
int a = 1, b = 2;
int min, max;

if (a > b) {
    printf("Promenna a je vetsi nez b");
    min = b;
    max = a;
} else  {
    printf("Promenna a neni vetsi nez b");
    min = a;
    max = b;
}
  \end{minted}
\end{frame}


\begin{frame}[t,fragile]\frametitle{Složitější příklad bez bloků} 
  \begin{minted}[linenos=true]{c} 
int a = 1, b = 2;
int min, max;

if (a > b) 
    printf("Promenna a je vetsi nez b");
    min = b;
    max = a;
  \end{minted}
\end{frame}



\begin{frame}[t,fragile]\frametitle{Líné vyhodnocování} 
  \begin{itemizex}
    \item Logické výrazy se nemusí vyhodnotit celé.
    \item Pokud je výsledek jasný, vyhodnocování nepokračuje.
  \end{itemizex}
  \begin{minted}[linenos=true]{c} 
int cislo = 0;

if (3 < 4 && (cislo = 5)) 
    printf("Cislo: %d", cislo); // Cislo: 5

if (3 < 4 || (cislo = 10))
    printf("Cislo: %d", cislo); // Cislo: 5

  \end{minted}
\end{frame}


\begin{frame}[t,fragile]\frametitle{Odpudivý příklad} 
  \begin{itemizex}
    \item Úkol: načtěte číslici a vypište na obrazovku její slovní název.
    \item $1\longrightarrow$ jedna, $2\longrightarrow$ dva, atd.
  \end{itemizex}

  \begin{minted}[linenos=true]{c} 
int cislo = 2;
if (cislo == 1)
    printf("jedna");
if (cislo == 2)
    printf("dva");
if (cislo == 3)
    printf("tri");
if (cislo == 4)
    printf("ctyri");
...
  \end{minted}
\end{frame}


\begin{frame}[t,fragile]\frametitle{Konstrukce \texttt{switch}} 
  \begin{itemizex}
    \item Funguje podobně jako hromada \texttt{if}ů za sebou.
    \item Proměnná \texttt{cislo} v příkladu musí být celé číslo.
  \end{itemizex}
  \begin{minted}[linenos=true]{c} 
int cislo = 2;
switch(cislo) {
    case 0: printf("nula"); break;
    case 1: printf("jedna"); break;
    case 2: printf("dva"); break;
    ...
    case 9: printf("devet"); break;
    default: printf("nevim"); break;
}
// dva
  \end{minted}
\end{frame}



\begin{frame}[t,fragile]\frametitle{Switch bez breaku} 
\begin{itemizex}
  \item Bez příkazu \texttt{break} pokračuje výpočet v další větvi.
\end{itemizex}
\begin{minted}[linenos=true]{c} 
int a = 0;
switch(a) {
    case 0: printf("nula ");
    case 1: printf("jedna ");
    case 2: printf("dva "); break;
    ...
    case 9: printf("devet "); break;
    default: printf("nevim ");
}

// nula jedna dva 
\end{minted}
\end{frame}


\begin{frame}[t,fragile]\frametitle{Switch s více příkazy} 
  \begin{minted}[linenos=true]{c} 
switch(a) {
    case 0: {
        printf("nula");
        cislo = 2 * a;
        printf("%d\n", cislo);
    }
    case 1: 
        printf("jedna ");
        cislo = 4 * a;
        printf("%d\n", cislo);
        break;
    case 2: printf("dva "); break;
    default: printf("nevim ");
}
  \end{minted}
\end{frame}


\begin{frame}[t,fragile]\frametitle{Switch se znaky} 
  \begin{itemize}
    \item \texttt{Switch} můžeme používat jen pro celá čísla.
    \item Ale znaky jsou ve skutečnosti celá čísla, hurá!
  \end{itemize}
  \begin{minted}[linenos=true]{c} 
char znak = '*';

switch(znak) {
    case 'a': printf("Prvni znak abecedy\n"); break;
    case 'b': printf("Druhy znak abecedy\n"); break;
    case 'c': printf("Treti znak abecedy\n"); break;
    case '*': printf("To je hvezdicka!\n"); break;
    default: printf("Dal uz to neumim\n"); break;
}
// To je hvezdicka!
  \end{minted}
\end{frame}



\begin{frame}[t,fragile]\frametitle{Znaky se opravdu chovají jako čísla} 
  \begin{minted}[linenos=true]{c} 
int x;
printf("%d", 'a'); // 97
printf("%d", 'z'); // 122
x = 'a' < 100; // x = 1
x = 'a' < 'z'; // x = 1
x = 'a' < ('z' / 2); // x = 0
  \end{minted}
\end{frame}


\begin{frame}[t,fragile]{Úlohy}
\begin{center}
\vskip 1cm
{\Large Seznam úloh je na \url{http://KMIup1.jdem.cz}}
\vskip 2cm
\url{http://tux.inf.upol.cz/~havrlant/}
\end{center}
\end{frame}


\end{document}